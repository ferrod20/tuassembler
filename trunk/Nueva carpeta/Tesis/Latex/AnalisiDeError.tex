\subsection{An�lisis de error} 
Para mejorar un modelo computacional necesitamos entender donde est� funcionando mal. Analizando el error en un clasificador como un eitquetador gramatical se realiza mediante un matriz de confusi�n o tablade contingencia. Una matriz de confusi�n para una tarea de clasificaci�n de N-vias es una matriz de NxN donde la celda $(x,y)$ contiene el n�mero de veces que un item con correcta clasigicaci�n x fu� clasuficado por el modelo como y. POr ejemplo, la siguiente tabla muestra una porci�n de la matriz de confusi�n para los experimentos de etiquetado con HMM de Franz. Las etiquetas de la fila indican las etiquetas correctas, las etiquetas de las columnas indican las etiquetas asignadas por el etiquetador, y cada celda indica el porcentaje del error de etiquetado general. Por lo tanto 4.4 del total de errores fueron causados por fallida etiquetacion de VBD como VBN. 
TABLA
La matriz anterior y el analisis de error relacionado en Franz, Kupiec y Ratnaparkhi sugieren que algunos de los mayores problemas que encaran los etiquetadores actuales son:
1( NN contra NNp contra JJ: 
\subsection{Primer experimento} 
El primer experimento consiste en medir (generando una matriz de confusi�n) la informaci�n extra�da de COBUILD contra la misma informaci�n generada a partir de un etiquetador autom�tico (TnT). Es decir, la informaci�n extra�da de COBUILD, como se mencion� anteriormente, es la uni�n de definiciones y ejemplos, con la informaci�n gramatical correspondiente a la palabra definida.
A continuaci�n se presenta un peque�o extracto:\\
 
\begin{multicols}{2} 
\noindent A\\
cat	NN\\
is\\
a\\
small\\
furry\\
animal\\
with\\
a\\
tail\\
,\\
whiskers\\
,\\
and\\
sharp\\
claws\\
that\\
kills\\
smaller\\
animals\\
such\\
as\\
mice\\
and\\
birds\\
.\\
Cats	NNS\\
are\\
often\\
kept\\
as\\
pets\\
.\\
She\\
put\\
out\\
a\\
hand\\
and\\
stroked\\
the\\
cat	NN\\
softly\\
...\\
...\\
domestic\\
animals\\
such\\
as\\
dogs\\
and\\
cats	NNS\\
.\\
\end{multicols}
Esta es la informaci�n extra�da de COBUILD para la palabra \textsl{cat}; la uni�n de la definici�n:\\
\\
\textsl{A cat is a small furry animal with a tail, whiskers, and sharp claws that kills smaller animals such as mice and birds. Cats are often kept as pets.}\\
\\
y los ejemplos\\
\\
\textsl{She put out a hand and stroked the cat softly... }\\
\textsl{...domestic animals such as dogs and cats.}\\
\\
Se puede notar la informaci�n gramatical expresada mediante las etiquetas NN y NNS para las palabras \textsl{cat} y \textsl{cats} respectivamente.
La idea de este experimento ser� comparar estas etiquetas contra las etiquetas asignadas por el etiquetador autom�tico TnT. Entonces se tomar� este corpus plano (sin etiquetas), se lo etiquetar� utilizando TnT entrenado con el corpus de entrenamiento Wall Street Journal (de ahora en m�s WSJ) \footnote{Wall Street Journal es un corpus anotado, parte del Penn Treebank} y luego se realizar� la comparaci�n.\\
La matriz de confusi�n\footnote{Las matrices de confusi�n presentadas de aqu� en adelante contienen las primeras 10 etiquetas de mayor error} generada a partir de dicha comparaci�n es la siguiente:

\begin{center}
\begin{longtable}{| l | c | c | c | c | c | c | c | c | c | c | }
\caption{Matriz de confusi�n para etiquetas extra�das de COBUILD vs generadas por TnT}\\	
\hline
 \backslashbox{\scriptsize{COBUILD}\kern-1em}{\kern-1em \scriptsize{TnT}}  &	\textbf{NN}	&   \textbf{VB}	&   \textbf{JJ}	&   \textbf{VBN}	&   \textbf{RB}	&   \textbf{VBG}	&   \textbf{NNP}	&   \textbf{IN}	&   \textbf{VBZ}	&   \textbf{NNS}	&   \hline
\endhead
\hline
\endfoot
\endlastfoot
	\hline
\textbf{NN} & - & \textbf{556} & \textbf{1953} & 52 & 86 & 276 & - & 8 & - & -\\
\textbf{VB} & \textbf{2616} & - & \textbf{614} & - & 42 & - & 77 & 15 & - & 5\\
\textbf{JJ} & \textbf{1577} & 96 & - & \textbf{1361} & \textbf{634} & \textbf{555} & \textbf{281} & 30 & - & 16\\
\textbf{VBN} & - & - & - & - & - & - & - & - & - & -\\
\textbf{RB} & 219 & 23 & \textbf{408} & 10 & - & 9 & 34 & 249 & - & 11\\
\textbf{VBG} & - & - & - & - & - & - & - & - & - & -\\
\textbf{NNP} & - & - & - & - & - & - & - & - & - & -\\
\textbf{IN} & - & - & - & - & - & - & - & - & - & -\\
\textbf{VBZ} & - & - & - & - & - & - & - & - & - & -\\
\textbf{NNS} & 83 & 1 & 17 & - & 1 & 2 & 104 & 3 & 192 & -\\
\hline
\end{longtable}
\end{center}

\noindent Porcentaje de aciertos: 99,16\% \\
\noindent Cantidad de errores: 13082\\

Se puede apreciar un alto porcentaje de aciertos entre las etiquetas extra�das de COBUILD (99,16\%) y las etiquetas asignadas por TnT. Este porcentaje indica que la informaci�n de etiquetas extra�das de COBUILD es consistente con las producidas por TnT. La mayor�a de los errores se da en etiquetas VB, NN y JJ de COBUILD cuando son etiquetadas como NN, JJ y NN por TnT respectivamente.

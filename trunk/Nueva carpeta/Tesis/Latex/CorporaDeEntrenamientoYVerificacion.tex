\section{Corpora de entrenamiento y corpora de verificaci�n} 
Los etiquetadores gramaticales que se basan en modelos de aprendizaje poseen un proceso de entrenamiento sobre un corpus etiquetado previamente en el cual se generan las probabilidades que se utilizan para tomar decisiones frente a palabras ambig�as. 

Dicho corpus de entrenamiento necesita ser cuidadosamente considerado: si es muy espec�fico al dominio, corpora pertenecientes a ese dominio ser�n etiquetados con  precisi�n, pero corpora de diferente dominio ser�n etiquetados con errores. Por otro lado, si el corpus de entrenamiento es muy general las probabilidades no alcanzar�n a reflejar el dominio.

Supongamos que estamos intentando etiquetar una oraci�n particular. Si nuestra oraci�n es parte del corpus de entrenamiento, las probabilidades de las etiquetas para esa oraci�n van a ser extraordinariamente precisas y vamos a sobreestimar la precisi�n de nuestro etiquetador. Se desprende como conclusi�n que el corpus de entrenamiento no debe ser parcial incluyendo esa oraci�n. Por lo tanto al trabajar con etiquetadores basados en modelos estoc�sticos, dado un corpus de datos relevante, es una tarea habitual dividir los datos en un corpus de entrenamiento y un corpus de verificaci�n. 

Una vez realizada esta divisi�n se entrena el etiquetador con el corpus de entrenamiento, se ejecuta el proceso de etiquetaci�n y luego se comparan los resultados con el corpus de verificaci�n.

En general existen dos m�todos para entrenar y verificar un etiquetador gramatical. En el primer m�todo, se divide el corpus disponible en tres partes: un corpus de entrenamiento, un corpus de verificaci�n y un corpus de test de desarrollo\footnote{Tambi�n llamado \textsl{devtest}}. Se entrena el etiquetador con el corpus de entrenamiento. Entonces se utiliza el corpus de test de desarrollo para eventualmente afinar o ajustar algunos par�metros y en general decidir cual es el mejor modelo. Una vez que se elige el supuesto mejor modelo, se corre contra el corpus de verificaci�n para analizar su rendimiento.

En el segundo m�todo de entrenamiento y verificaci�n, se elige aleatoreamente una divisi�n de corpus de entrenamiento y verificaci�n para nuestros datos. Se entrena el etiquetador y luego se calcula el error en el corpus de verificaci�n. A continuaci�n se repite con un corpus de entrenamiento y de  verificacion diferente seleccionado aleatoreamente. La repetici�n de este proceso, llamado validaci�n cruzada, generalmente es realizada 10 veces. Luego se promedian esas 10 corridas para obtener un promedio en la proporci�n del error.
\subsection{Corpora de entrenamiento y corpora de verificaci�n} 
En los etiquetadores gramaticales que se basan en modelos estoc�sticos, las probabilidades se obtienen de un corpus en el que se han entrenado. 

Este corpus de entrenamiento necesita ser cuidadosamente considerado. Si el corpus de entrenamiento es muy espec�fico al dominio, las probailidades van a ser muy estrechas y no generalizar�n bien la etiquetaci�n de oraciones en diferentes dominios. Pero si el corpus de entrenamiento es muy general, estas probabilidades no van a hacer el trabajo suficiente de reflejar eldominio.

Supongamos que estamos intentando etiquetar una oraci�n de particular. Si nuestra oraci�n es parte del corpus de entrenamiento, las probabilidades de las etiquetas para esa oraci�n van a ser extraordinariamente precisas y vamos a sobreestimar la precisi�n de nuestro etiquetador. Se desprende como conclusi�n que el corpus de entrenamiento no debe ser parcial incluyendo esa oraci�n. Por lo tanto al trabajar con etiquetadores basados en modelos estoc�sticos, dado un corpus de datos relevante, es una tarea habitual dividir los datos en un corpus de entrenamiento y un corpus de verificaci�n. 

Luego se entrena el etiquetador con el corpus de entrenamiento y como pr�ximo paso se ejecuta el proceso de etiquetaci�n y se comparan los resultados con el corpus de verificaci�n.

En general existen dos m�todos para entrenar y verificar un etiquetador gramatical. En el primer m�todo, se divide el corpus disponible en tres partes: un corpus de entrenamiento, un corpus de test y un corpus de test de desarrollo. Se entrena el etiquetador con el corpus de entrenamiento. Entonces se utiliza el conjunto de test de desarrollo (tambi�n llamado devtest) para eventualmente afinar o ajustar algunos par�metros y en general decidir cual es el mejor modelo. Una vez que se elige el supuesto mejor modelo, se corre en el corpus de verificaci�n para ver su rendimiento.

En el segundo m�todo de entrenamiento y verificaci�n, se elige aleatoreamente una divisi�n de corpus de entrenamiento y verificaci�n para nuestros datos. Se entrena el etiquetador y luego se computa la proporci�n de error en el corpus de verificaci�n. A continuaci�n se repite con un conjunto de entrenamiento y de  verificacion diferente seleccionado aleatoreamente. 

La repetici�n de este proceso, llamado validaci�n cruzada generalmente es realizada 10 veces. Luego se promedian esas 10 corridas para obtener un promedio n la proporci�n del error.

Al comparar modelos es importante utilizar verificaciones estad�sticas para determinar si la diferencia entre los modelos es significantiva.

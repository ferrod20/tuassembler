\section{Introducci�n} 
El etiquetado o anotado gramatical, tambi�n conocido como Part-of-speech tagging, POS tagging o simplemente POST, es el proceso de asignar una etiqueta gramatical a cada una de las palabras de un texto seg�n su categor�a l�xica. Por ejemplo tomemos la oraci�n siguiente:\\

\noindent \textsl{There is no asbestos in our products now.} \\

\noindent El resultado de etiquetarla gramaticalmente es: \\

\noindent \textsl{There/EX is/VBZ no/DT asbestos/NN in/IN our/PRP products/NNS now/RB ./.}\\

\noindent donde cada palabra est� sucedida por una barra oblicua seguida de la etiqueta gramataical asignada. 

Un tagger o etiquetador es un programa que realiza este proceso autom�ticamente. La mayor�a de los taggers actuales utilizan modelos estad�sticos. Estos modelos se nutren entrenando el tagger con un texto anotado previamente (corpus de entrenamiento). El rendimiento del tagger (que se mide por las etiquetas asignadas correctamente) es fuertemente dependiente del corpus de entrenamiento utilizado. 

El problema reside en que la generaci�n de corpus de entrenamiento es una tarea muy costosa, por lo tanto la cantidad y calidad de los mismos es limitada. La idea central de esta tesis es la de suplir la falta de corpus de entrenamiento generando una nueva fuente de informaci�n a partir de una fuente de informaci�n existente: un diccionario.

Generalmente un diccionario contiene la definici�n de una palabra, una explicaci�n de su significado, algunas caracter�sticas como su pronunciaci�n y particularmente su clase gramatical y uno o m�s ejemplos que muestran su uso. Por lo tanto, extrayendo todos los ejemplos de un diccionario, se puede generar un corpus anotado parcialmente, es decir, un conjunto de oraciones donde alguna/s de las palabras que comprenden cada oraci�n posee/n una etiqueta gramatical.

\section{Descripci�n de la propuesta} 
El objetivo de este trabajo como se mencion� anteriormente es generar una nueva fuente de informaci�n a partir de un diccionario y luego utilizarla como corpus de entrenamiento intentando mejorar el rendimiento de un tagger.
\\
\\
\noindent El diccionario elegido para el trabajo es Cobuild: Cobuild es un diccionario de la lengua inglesa basado en la informaci�n del corpus Bank of English y el corpus Collins. Todos los ejemplos del diccionario Cobuild muestran patrones gramaticales t�picos, vocabulario t�pico y contextos t�picos para cada palabra. En consecuencia, Cobuild presenta una cantidad exhaustiva del vocabulario ingl�s derivado de observaciones directas del lenguaje.
\\
\\
\subsection{Primer etapa} 
La primer etapa consiste en extraer cuidadosamente los ejemplos (que son oraciones completas conteniendo la palabra definida con informaci�n l�xica) del diccionario Cobuild y generar a partir de estos un corpus parcialmente anotado. Luego se completar� el etiquetado del corpus utilizando un etiquetador autom�tico.
\\
\\
\subsection{Segunda etapa} 
La segunda etapa consiste en entrenar los taggers con este nuevo corpus de entrenamiento, etiquetar y luego medir los resultados.

Los taggers que se utilizar�n para el entrenamiento y medici�n son: 

\begin{itemize}
	\item Trigrams'n'Tags tagger (estoc�stico) 
	\item Stanford tagger (tagger de m�xima entrop�a).
\end{itemize}


Los corpus elegidos para realizar la medici�n son: 

\begin{itemize}
	\item Wall Street Journal
	\item British National Corpus
\end{itemize}
Ambos pertenecientes a la lengua inglesa (americana y brit�nica respectivamente).

British National Corpus es una colecci�n de 100 millones de palabras provenientes de textos escritos e ingl�s hablado creado en los 1990s por un consorcio de editores, universidades (Oxford y Lancaster) y la British Library.

Wall Street Journal es un corpus anotado, parte del Penn Treebank, de algo m�s de 1 mill�n de palabras.
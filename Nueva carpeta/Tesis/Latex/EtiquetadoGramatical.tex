\section{Etiquetado gramatical}
\noindent Como se mencion� anteriormente, el etiquetado gramatical es el proceso de asignar una etiqueta a cada una de las palabras de un texto seg�n su categor�a l�xica. O expresado de otra manera: dada una secuencia de palabras como entrada, obtener como salida una secuencia de etiquetas para las mismas. Este proceso se realiza en base a la definici�n de la palabra y al contexto en que �sta aparece. \\

\noindent Por ejemplo en: \\
\\
\textsl{Does that flight serve dinner} \\

\noindent \textsl{dinner} es un sustantivo y por lo tanto recibe la etiqueta para sustantivos NN.\\

El etiquetado gramatical brinda una gran cantidad de informaci�n sobre una palabra y sus vecinas. Por ejemplo, las etiquetas distinguen entre pronombres posesivos (mi, tu, su, �tc.) y pronombres personales (Yo, T�, �l, �tc.). Saber si una palabra es un pronombre posesivo o personal nos brinda informaci�n sobre las palabras que pueden ocurrir a continuaci�n: los pronombres posesivos generalmente son sucedidos por un sustantivo (como en \textsl{Mi comida}) mientras que los personales son sucedios por un verbo (como en \textsl{Yo duermo}). 

Utilizando esta deducci�n podemos aseverar que si una palabra fu� etiquetada como pronombre personal, es muy probable que la pr�xima palabra sea un verbo. 
Una etiqueta gramatical tambi�n puede ofrecer informaci�n relacionada con la pronunciaci�n de la palabra. En ingl�s la palabra \textsl{content} puede ser un sustantivo o un adjetivo y su pronunciaci�n var�a dependiendo de este hecho. Utilizando estas ideas se pueden producir pronunciaciones m�s naturales en un sistema de s�ntesis del habla (texto a voz) o tambi�n se puede obtener m�s exactitud en un sistema de reconocimiento del habla (voz a texto).

Otra aplicaci�n importante del etiquetado gramatical en sistemas de recuperaci�n de la informaci�n es el reconocimiento de sustantivos u otro tipo de palabras importantes dentro de un documento, para guardar y utilizar esta informaci�n en b�squedas posteriores. 

Por �ltimo, la asignaci�n autom�tica de etiquetas gramaticales juega un papel importante en algoritmos de desambig�aci�n del sentido de la palabra y en modelos ling��sticos basados en n-gramas utilizados en sistemas de reconocimiento del habla.

\subsection{Etiquetado gramatical}
\noindent Como se mencion� anteriormente, el etiquetado gramatical, tambi�n conocido como Part-of-speech tagging, POS tagging o simplemente POST, es el proceso de asignar una etiqueta a cada una de las palabras de un texto seg�n su categor�a l�xica. Este proceso se realiza en base a la definici�n de la palabra y la de sus palabras vecinas, es decir, el contexto en que �sta aparece. Por ejemplo en \textsl{Does that flight serve dinner}, \textsl{dinner} es un sustantivo y por lo tanto recibe la etiqueta para sustantivos \textsl{NN}.

El etiquetado gramatical brinda una gran cantidad de informaci�n sobre una palabra y sus vecinas. Por ejemplo, las etiquetas distinguen entre pronombres posesivos (mi, tu, su, �tc.) y pronombres personales (Yo, T�, �l, �tc.). Saber si una palabra es un pronombre posesivo o personal nos brinda informaci�n sobre las palabras que pueden ocurrir a continuaci�n: los pronombres posesivos generalmente son sucedidos por un sustantivo (como en \textsl{Mi comida}) mientras que los personales son sucedios por un verbo (como en \textsl{Yo duermo}). 

Utilizando esta deducci�n podemos aseverar que si una palabra fu� etiquetada como pronombre personal, es muy probable que la pr�xima palabra sea un verbo. Este conocimiento puede ser de �til aplicaci�n en modelos lingu�sticos para reconocimiento del habla (voz a texto). Pero esta no es la �nica informaci�n que una etiqueta gramatical nos puede ofrecer. 

Una etiqueta gramatical tambi�n nos puede acercar informaci�n relacionada con la pronunciaci�n de la palabra. En ingl�s la palabra \textsl{content} puede ser un sustantivo o un adjetivo y su pronunciaci�n var�a dependiendo de este hecho. Utilizando estas ideas podemos producir pronunciaciones m�s naturales en un sistema de s�ntesis del habla (texto a voz) o tambi�n podemos obtener m�s exactitud en un sistema de reconocimiento del habla (voz a texto).

Otra aplicaci�n importante del etiquetado gramatical en sistemas de recuperaci�n de la informaci�n es el reconocimiento de sustantivos u otro tipo de palabras importantes dentro de un documento, para guardar y utilizar esta informaci�n en b�squedas posteriores. 

Por �ltimo, la asignaci�n autom�tica de etiquetas gramaticales juega un papel importante en algoritmos de desambiguaci�n del sentido de la palabra y en modelos ling��sticos basados en n-gramas utilizados en sistemas de reconocimiento del habla.

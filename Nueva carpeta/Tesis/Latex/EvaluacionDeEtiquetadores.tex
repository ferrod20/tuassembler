\subsection{Evaluaci�n de etiquetadores gramaticales} 
Los etiquetadores gramaticales generalmente son evaluados comparando su precisi�n contra un corpus de verificaci�n Gold Standard etiquetado por humanos. Por precisi�n nos referimos al porcentaje de todas las etiquetaas en el corpus de verificaci�n donde el etiquetador y el Gold stantard concuerdan. Los algoritmos mas corrientes de etiquetado gramatical tienen una precisi�n alrededor del 96-97 para corpus de etiquetas simples como el corpus del Penn Treebank. Estas precisiones son para palabras y puntuaciones, la precisi�n para palabras solamente es menor.
Qu� tan bueno es un 97? El rendimiento de la etiquetaci�n puede ser comparado contra un l�mite inferior o piso y un l�mite superior o techo. Una manera de establecer un techo es ver que tan bien realizan la tarea los humanos. 

Marcus, por ejemplo, encontr� que los etiquetadores humanos concuerdan alrededor del 96-97 de las etiquetas en la versi�n Penn Treebank del corpus Brown. Esto sugiere que el Gold Standard debe tener un 3-4 de margen de error, y por lo tanto no tiene sentido obtener una precisi�n del 100. Ratnaparkhi mostr� que en las palabras donde su etiquetador ha tenido problemas de ambiguedad de etiquetaci�n fueron exactamente las mismas en donde los humanos han etiquetado inconsistentemente en el corpus de entrenamiento. Dos experimientos por Voutilainen encontraron que cuando a los humanos se les permiti� discutir etiquetas, alcanzaron un consenso en el 100 de las etiquetas.

El piso standard sugerido por Gale es elegir la etiqueta m�s probable con unigrama para cada palabra ambigua. La etiqueta m�s probable para cada palabra puede ser computada desde un corpus etiquetado a mano (que puede ser el mismo que el corpus de entrenamiento para el etiquetador que est� siendo evaluado).

\section{Evaluaci�n de etiquetadores gramaticales} 
Los etiquetadores gramaticales generalmente son evaluados comparando su precisi�n contra un corpus de verificaci�n\footnote{Tambi�n llamado \textsl{Gold Standard}} etiquetado por humanos. Definimos precisi�n como el porcentaje de todas las etiquetas en el corpus de verificaci�n donde el etiquetador y el \textsl{Gold Standard} concuerdan. Los algoritmos actuales de etiquetado gramatical tienen una precisi�n del 96\%-97\% para conjuntos de etiquetas simples como el \textsl{Penn Treebank}. Estas precisiones son para palabras y puntuaciones, la precisi�n para palabras solas es menor.

Naturalmente uno tiende a preguntarse qu� tan bueno es un 97\%. El rendimiento de un proceso de etiquetado puede ser comparado contra un l�mite inferior y un l�mite superior. Una manera de establecer un l�mite superior es ver que tan bien realizan la tarea los humanos. 

\textsl{Marcus}, por ejemplo, encontr� que los etiquetadores humanos concuerdan en el 96\%-97\% de las etiquetas en el corpus \textsl{Brown} etiquetado con etiquetas \textsl{Penn Treebank}. Esto sugiere que el \textsl{Gold Standard} debe tener un 3\%-4\% de margen de error, y por lo tanto no tiene sentido obtener una precisi�n del 100\%. \textsl{Ratnaparkhi} mostr� que en las palabras donde su etiquetador ha tenido problemas de ambig�edad de etiquetaci�n fueron exactamente las mismas en donde los humanos han etiquetado inconsistentemente el corpus de entrenamiento. Dos experimientos realizados por \textsl{Voutilainen} encontraron que cuando a los humanos se les permiti� discutir etiquetas, alcanzaron un consenso en el 100\% de las etiquetas.

Por otro lado el l�mite inferior sugerido por \textsl{Gale} es elegir la etiqueta m�s probable aplicando el modelo de unigrama para cada palabra ambig�a. La etiqueta m�s probable para cada palabra puede ser computada desde un corpus etiquetado a mano (que puede ser el mismo que el corpus de entrenamiento para el etiquetador que est� siendo evaluado).

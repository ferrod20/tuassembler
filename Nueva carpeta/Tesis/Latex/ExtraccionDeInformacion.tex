\subsection{Extracci�n de la informaci�n} 
El diccionario COBUILD guarda su informaci�n en un archivo de texto plano con un formato particular. El primer desaf�o de este trabajo fu� comprender y extraer la informaci�n alamacenada en ese archivo. A continuaci�n se muestra un peque�o fragmento del mismo para ejemplificar

\begin{verbatim}
DICTIONARY_ENTRY
ace
aces
*e*!is    
If you are or come within an ace of something, you very nearly do or experience it.
He came within an ace of being run over.
phrase: verb inflects
phrase                                

DICTIONARY_ENTRY
ace
aces
*e*!is    
A person who is ace at something is extremely good at it; an informal use.
...an ace marksman.
classifying adjective
adjective                                

DICTIONARY_ENTRY
ace
aces
*e*!is    
If you say that something is ace, you mean that you think that it is very good; 
an informal use.
Their new records really ace!
qualitative adjective or exclamation
adjective                                
\end{verbatim}

Cada entrada arriba presentada tiene la caracter�stica de poseer una cantidad variable de campos y no es posible identificarlos exactamente. Sin embargo, contienen algunos rasgos comunes: la palabra, sus formas, la pronunciaci�n, su definici�n y uno o m�s ejemplos donde se indica como se emplea (mediante una etiqueta gramatical). Por ejemplo, en la primer entrada se pueden distinguir estos campos:

\begin{verbatim}
DICTIONARY_ENTRY
ace
aces
*e*!is    
If you are or come within an ace of something, you very nearly do or experience it.
He came within an ace of being run over.
phrase: verb inflects
phrase                                
\end{verbatim}

Estas entradas, que conforman el diccionario COBUILD y que constituyen la fuente de informaci�n principal sobre la cual se basa este trabajo, fueron cuidadosamente procesadas y refinadas intentando mantener toda la informaci�n disponible. El primer desaf�o de esta etapa consisti� en recuperar las entradas con toda la informaci�n gramatical disponible; expl�cita e impl�cita. Una primer tarea fu� reconocer y registrar informaci�n relacionada a las formas flexionadas de la palabra (plurales, pasados, etc.), es decir, obtener informaci�n gramatical impl�cita.

\subsubsection{Reconocimiento de formas flexionadas} 
En muchas entradas del diccionario COBUILD ocurre la palabra, uno o m�s ejemplos en donde �sta aparece con cierto sentido (indicado por medio de etiquetas gramaticales) pero dentro de los ejemplos hay apariciones de formas flexionadas. Tomemos la siguiente entrada:

\begin{verbatim}
DICTIONARY_ENTRY
bite
bites, biting, bit, bitten
b*a*!it    
If an object or surface bites, it grips another object or surface rather than slipping
on it or against it.
Let the clutch in slowly until it begins to bite.
verb
verb                                
\end{verbatim}

Aqu� arriba se puede observar una entrada del diccionario para la palabra \textsl{bite}, que contiene dos ejemplos de esta palabra con sus respectivas etiquetas:\\

(1) \textsl{If an object or surface bites, it grips another object or surface rather than slipping
on it or against it.}\\

(2) \textsl{Let the clutch in slowly until it begins to bite.}\\
\\
En (2) aparece la palabra \textsl{bite} en su forma regular con la etiqueta \textsl{verb} mientras que en (1) aparece la forma flexionada \textsl{bites} con la etiqueta \textsl{verb}. En este caso el ejemplo est� ofreciendo m�s informaci�n gramatical que la expuesta por medio de la etiqueta. Reconociendo la forma flexionada (\textsl{bites}) podemos adicionarle informaci�n extra a la etiqueta \textsl{verb}; en vez de guardar la etiqueta de Tree Bank correspondiente a \textsl{verb} (VB), en este caso guardar�amos la etiqueta  VBZ (verbo de tiempo presente en tercera persona singular) que contiene m�s informaci�n gramatical que VB.

Las entradas de COBUILD exponen las formas derivadas de la palabra que pueden contener los ejemplos. En el ejemplo presentado anteriormente la palabra es \textsl{bite} y las formas derivadas de \textsl{bite} que muestra la entrada son \textsl{bites, biting, bit} y \textsl{bitten}. Con esta informaci�n y la etiqueta que fu� anotada en COBUILD (\textsl{verb} en ambos casos) se puede inferir y generar etiquetas de Tree Bank con informaci�n adicional. Como ya se mencion� anteriormente, en este caso la forma \textsl{bites} (derivada de la palabra \textsl{bite}) que aparece en el primer ejemplo posee la etiqueta \textsl{verb}. La tarea aqu� ser� reconocer que \textsl{bites} es un verbo de tiempo presente en tercera persona singular a partir de que \textsl{bites} est� etiquetada como verbo y de que la palabra de la cual deriva es \textsl{bite}. Es decir, inferir el tipo de la forma derivada a partir de la palabra y la etiqueta asignada por COBUILD.

Con el objetivo de identificar las formas derivadas de una palabra se desarrollaron reglas y m�todos para su reconocimiento, buscando preservar y aprovechar toda  la informaci�n que ofrece COBUILD. Entonces, a partir de esta informaci�n: la palabra, la forma en que ocurre y la etiqueta asignada aplicamos las siguientes reglas para reconocer informaci�n adicional a la etiqueta gramatical.

\floatname{algorithm}{Algoritmo}
\begin{algorithm}
\caption{Reconocimiento de formas derivadas}
\begin{algorithmic}
\STATE Traducir la etiqueta asignada por Cobuild a PenTreeBank
\STATE Si la etiqueta obtenida es
\STATE
\STATE \textbf{JJ}:
\STATE\hspace{\algorithmicindent} Si la forma termina en \textsl{er} o empieza en \textsl{more} o \textsl{less} aplicar \textbf{JJR}
\STATE\hspace{\algorithmicindent} Si la forma termina en \textsl{est} o empieza en \textsl{most} o \textsl{least} aplicar \textbf{JJS}
\STATE
\STATE \textbf{RB}:
\STATE\hspace{\algorithmicindent}Si la forma termina en \textsl{er} o empieza en \textsl{more} o \textsl{less} aplicar \textbf{RBR}
\STATE\hspace{\algorithmicindent}Si la forma termina en \textsl{est} o empieza en \textsl{most} o \textsl{least} aplicar \textbf{RBS}
\STATE
\STATE \textbf{NN}:
\STATE\hspace{\algorithmicindent} Si la forma termina en \textsl{s} aplicar \textbf{NNS}
\STATE
\STATE \textbf{VB}:
\STATE\hspace{\algorithmicindent} Si la forma termina en \textsl{ed} aplicar \textbf{VBD|VBN}

\STATE\hspace{\algorithmicindent} Si la forma termina en \textsl{ing} aplicar \textbf{VBG}
\STATE\hspace{\algorithmicindent} Si la forma es igual a la palabra y la palabra anterior es \textsl{to} aplicar \textbf{VBP}
\STATE\hspace{\algorithmicindent} Si la forma termina en \textsl{s} aplicar \textbf{VBZ}
\end{algorithmic}
\end{algorithm}

Aplicando algoritmos de extracci�n y el algoritmo de reconocimiento de formas derivadas explicado anteriormente se obtiene un nuevo corpus parcialemente anotado a partir del diccionario Cobuild. A continuaci�n este corpus ser� procesado y utilizado como corpus de entrenamiento.
\subsection{Extracci�n de la informaci�n} 
El diccionario COBUILD guarda su informaci�n en un archivo de texto plano con un formato particular. El primer desaf�o de este trabajo fu� comprender y extraer la informaci�n alamacenada en ese archivo. A continuaci�n se muestra peque�o fragmento del mismo para ejemplificar

\begin{verbatim}
DICTIONARY_ENTRY
ace
aces
*e*!is    
If you are or come within an ace of something, you very nearly do or experience it.
He came within an ace of being run over.
phrase: verb inflects
phrase                                
DI000183
004

DICTIONARY_ENTRY
ace
aces
*e*!is    
A person who is ace at something is extremely good at it; an informal use.
...an ace marksman.
classifying adjective
adjective                                
DI000183
005    
expert

DICTIONARY_ENTRY
ace
aces
*e*!is    
If you say that something is ace, you mean that you think that it is very good; 
an informal use.
Their new records really ace!
qualitative adjective or exclamation
adjective                                
DI000183
006    
great
lousy               
\end{verbatim}

Cada entrada posee una cantidad variable de campos y no es posible identificarlos exactamente. Sin embargo contienen generalmente la palabra, sus formas y uno o m�s ejemplos donde se indica como se emplea (mediante una etiqueta gramatical). Estas entradas, que constituyen la informaci�n inicial en la que se basa este trabajo y que conforman el diccionario COBUILD, fueron cuidadosamente procesadas y refinadas intentando mantener toda la informaci�n disponible. 

El primer desaf�o de esta etapa consisti� en reconocer y registrar informaci�n relacionada a las formas flexionadas de la palabra (plurales, pasados, etc.). En muchas entradas ocurre la palabra, uno o m�s ejemplos en d�nde esta aparece con su correspondiente uso (indicado por medio de etiquetas gramaticales) pero las apariciones dentro de los ejemplos ocurren con formas flexionadas. Tomemos como ejemplo esta entrada:

\begin{verbatim}
DICTIONARY_ENTRY
bite
bites, biting, bit, bitten
b*a*!it    
If an object or surface bites, it grips another object or surface rather than slipping
on it or against it.
Let the clutch in slowly until it begins to bite.
verb
verb                                
DI002405
009    
catch  
grip
\end{verbatim}

Aqu� arriba podemos observar una entrada del diccionario para la palabra \textsl{bite}, conteniendo dos ejemplos de esta palabra con sus respectivas etiquetas gramaticales. Podemos notar que en el primer ejemplo la palabra aparece en la forma flexionada; \textsl{bites}. Es decir que la entrada del diccionario nos est� ofreciendo m�s informaci�n que la que se observa a simple vista. Se puede observar que en el primer ejemplo la palabra aparece en la forma flexionada \textsl{bites} y la entrada asigna la etiqueta \textsl{verb} mientras que en el segundo ejemplo la palabra aparece en su forma regular \textsl{bite} con la misma etiqueta. Reconociendo la forma flexionada (\textsl{bites}) podemos adicionarle informaci�n extra a la etiqueta \textsl{verb}, es decir que aparte de guardar la etiqueta \textsl{verb} podemos guardar la informaci�n de que ocurre en una forma flexionada.

Las entradas de COBUILD exponen las formas derivadas de la palabra que pueden contener los ejemplos. En el ejemplo presentado anteriormente la palabra es \textsl{bite} y las formas derivadas de \textsl{bite} que muestra la entrada son \textsl{bites, biting, bit} y \textsl{bitten}. Con esta informaci�n y la etiqueta que fu� anotada en COBUILD (\textsl{verb} en ambos casos) podemos inferir y generar informaci�n adicional para las etiquetas asignadas. Como ya mencionamos, en este caso la forma \textsl{bites} (derivada de la palabra \textsl{bite}) que aparece en el primer ejemplo posee la etiqueta \textsl{verb}. La tarea aqu� ser� reconocer que \textsl{bites} es un verbo flexionado a partir de que \textsl{bites} est� etiquetada como verbo y de que la palabra de la cual deriva es \textsl{bite}. Es decir, inferir el tipo de la forma derivada a partir de la palabra y la etiqueta asignada.

Con el objetivo de identificar las formas derivadas de una palabra se desarrollaron reglas y m�todos para su reconocimiento, buscando preservar y aprovechar toda  la informaci�n que ofrece COBUILD. 
Entonces, a partir de esta informaci�n: la palabra, la forma en que ocurre y la etiqueta asignada aplicamos las siguientes reglas para agregar informaci�n adicional a la etiqueta gramatical.

\floatname{algorithm}{Algoritmo}
\begin{algorithm}
\caption{Reconocimiento de formas derivadas}
\begin{algorithmic}
\STATE Traducir la etiqueta asignada por Cobuild a PenTreeBank
\STATE Si la etiqueta obtenida es
\STATE
\STATE \textbf{JJ}:
\STATE\hspace{\algorithmicindent} Si la forma termina en \textsl{er} o empieza en \textsl{more} o \textsl{less} aplicar \textbf{JJR}
\STATE\hspace{\algorithmicindent} Si la forma termina en \textsl{est} o empieza en \textsl{most} o \textsl{least} aplicar \textbf{JJS}
\STATE
\STATE \textbf{RB}:
\STATE\hspace{\algorithmicindent}Si la forma termina en \textsl{er} o empieza en \textsl{more} o \textsl{less} aplicar \textbf{RBR}
\STATE\hspace{\algorithmicindent}Si la forma termina en \textsl{est} o empieza en \textsl{most} o \textsl{least} aplicar \textbf{RBS}
\STATE
\STATE \textbf{NN}:
\STATE\hspace{\algorithmicindent} Si la forma termina en \textsl{s} aplicar \textbf{NNS}
\STATE
\STATE \textbf{VB}:
\STATE\hspace{\algorithmicindent} Si la forma termina en \textsl{ed} aplicar \textbf{VBD|VBN}

\STATE\hspace{\algorithmicindent} Si la forma termina en \textsl{ing} aplicar \textbf{VBG}
\STATE\hspace{\algorithmicindent} Si la forma es igual a la palabra y la palabra anterior es \textsl{to} aplicar \textbf{VBP}
\STATE\hspace{\algorithmicindent} Si la forma termina en \textsl{s} aplicar \textbf{VBZ}
\end{algorithmic}
\end{algorithm}

Aplicando el proceso de extracci�n y reconocimiento de formas derivadas explicado anteriormente se obtiene a partir del diccionario Cobuild un nuevo corpus parcialemente anotado. A continuaci�n este corpus ser� procesado y utilizado como corpus de entrenamiento.
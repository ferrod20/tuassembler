\begin{thebibliography}{99}
\bibitem{Jurafsky} Jurafsky, D. \& Martin, J. H., Speech and Language Processing: An introduction to natural language processing, computational linguistics, and speech recognition, Second edition, chapter 5, New Jersey: Prentice Hall.
\bibitem{LibroManningSchutze} Chris Manning and Hinrich Sch�tze, Foundations of Statistical Natural Language Processing, MIT Press. Cambridge, MA: May 1999
\bibitem{TnT}Brants, T. (2000). TnT: a statistical part-of-speech tagger. In Proceedings of the sixth conference on Applied natural language processing, p.224-231, April 29-May 04, 2000, Seattle, Washington. Morgan Kaufmann Publishers Inc.
\bibitem{StanfordTagger} Kristina Toutanova and Christopher D. Manning. 2000. Enriching the Knowledge Sources Used in a Maximum Entropy Part-of-Speech Tagger. In Proceedings of the Joint SIGDAT Conference on Empirical Methods in Natural Language Processing and Very Large Corpora (EMNLP/VLC-2000), pp. 63-70.
\bibitem{PennTreebank} Mitchell P. Marcus , Mary Ann Marcinkiewicz , Beatrice Santorini, Building a large annotated corpus of English: the penn treebank, Computational Linguistics, v.19 n.2, June 1993
\bibitem{TagMapping} Stevenson	M., A corpus-based approach to deriving lexical mappings, EACL '99 Proceedings of the ninth conference on European chapter of the Association for Computational Linguistics, Pages 285-286

%Palabras desconocidas
\bibitem{BaayenAndSproat}Baayen, H. and Sproat, R. (1996). Estimating lexical priors for low-frequency morphologically ambiguous forms. Computational Linguistics, 22(2), 155-166.
\bibitem{DermatasAndKokkinakis}Dermatas, E. and Kokkinakis, G. (1995). Automatic stochastic tagging of natural language texts. Computational Linguistics, 21(2), 137-164
\bibitem{Weischedel}Weischedel, R., Meteer, M., Schwartz, R., Ramshaw, L. A., and Palmucci, J. (1993). Coping with ambiguity and unknown words through probabilistic models. Computational Linguistics, 19(2), 359-382.
\bibitem{Samuelsson}Samuelsson, C. (1993). Morphological tagging based entirely on Bayesian inference. In 9th Nordic Conference on Computational Linguistics NODALIDA-93. Stockholm.

%Analisis de error
\bibitem{Kupiec}Kupiec, J. (1992). Robust part-of-speech tagging using a hidden Markov model. Computer Speech and Language, 6, 225-242.
\bibitem{Franz}Franz, A. (1996). Automatic Ambiguity Resolution in Natural Language Processing. Springer-Verlag, Berlin.
%Evaluaci�n de taggers
\bibitem{Marcus}Marcus, M. P., Santorini, B., and Marcinkiewicz, M. A. (1993). Building a large annotated corpus of English: The Penn treebank. Computational Linguistics, 19(2), 313-33
\bibitem{Ratnaparkhi}Ratnaparkhi, A. (1996). A maximum entropy part-of-speech tagger. In Proceedings of the Conference on Empirical Methods in Natural Language Processing, University of Pennsylvania, pp. 133-142. ACL
\bibitem{Voutilainen}Voutilainen, A. (1995). Morphological disambiguation. In Karlsson, F., Voutilainen, A., Heikkil�, J., and Anttila, A.(Eds.), Constraint Grammar: A Language Independent System forParsing Unrestricted Text, pp. 165-284. Mouton deGruyter, Berlin.
\bibitem{Gale}Gale, W. A., Church, K. W., and Yarowsky, D. (1992). Estimating upper and lower bounds on the performance of word-sense disambiguation programs. In Proceedings of the 30th ACL, Newark, DE, pp. 249-256. ACL.
%Def Corpus
\bibitem{DefCorpus1} Brown K. (Editor) 2005. Encyclopedia of Language and Linguistics - 2nd Edition. Oxford: Elsevier.
\bibitem{AnotacionesMinimas} Sinclair, J. 'The automatic analysis of corpora', in Svartvik, J. (ed.) Directions in Corpus Linguistics (Proceedings of Nobel Symposium 82). Berlin: Mouton de Gruyter. 1992.
\bibitem{AnotacionesTodas}Wallis, S. 'Annotation, Retrieval and Experimentation', in Meurman-Solin, A. \& Nurmi, A.A. (ed.) Annotating Variation and Change. Helsinki: Varieng, [University of Helsinki]. 2007
\bibitem{BNC} Guy Aston and Lou Burnard, 1998. The BNC Handbook: Exploring the British National Corpus with SARA, Edinburgh University Press

%Corpus
\bibitem{DescEnglishUsage} Quirk, R. 'Towards a description of English Usage', Transactions of the Philological Society. 1960. 40-61.
\bibitem{MotrealFrenchProject}Sankoff, D. \& Sankoff, G. Sample survey methods and computer-assisted analysis in the study of grammatical variation. In Darnell R. (ed.) Canadian Languages in their Social Context Edmonton: Linguistic Research Incorporated. 1973. 7-64.
\bibitem{Poplack}Poplack, S. The care and handling of a mega-corpus. In Fasold, R. \& Schiffrin D. (eds.) Language Change and Variation, Amsterdam: Benjamins. 1989. 411-451.
\bibitem{Andersen1}Andersen, Francis I.; Forbes, A. Dean (2003), "Hebrew Grammar Visualized: I. Syntax", Ancient Near Eastern Studies 40: 43-61
\bibitem{Andersen2}Eyland, E. Ann (1987), "Revelations from Word Counts", in Newing, Edward G.; Conrad, Edgar W., Perspectives on Language and Text: Essays and Poems in Honor of Francis I. Andersen's Sixtieth Birthday, July 28, 1985, Winona Lake, IN: Eisenbrauns, p. 51, ISBN 0-931464-26-9
\bibitem{Quaric}Dukes, K., Atwell, E. and Habash, N. 'Supervised Collaboration for Syntactic Annotation of Quranic Arabic'. Language Resources and Evaluation Journal. 2011.
\bibitem{WallisAndNelson}Wallis, S. and Nelson G. 'Knowledge discovery in grammatically analysed corpora'. Data Mining and Knowledge Discovery, 5: 307-340. 2001.
\end{thebibliography}
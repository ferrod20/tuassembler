\begin{thebibliography}{99}
\bibitem{Jurafsky} Jurafsky, D. & Martin, J. H., Speech and Language Processing: An introduction to natural language processing, computational linguistics, and speech recognition, Second edition, chapter 5, New Jersey: Prentice Hall.
\bibitem{LibroManningSchutze} Chris Manning and Hinrich Sch�tze, Foundations of Statistical Natural Language Processing, MIT Press. Cambridge, MA: May 1999
\bibitem{TnT} Thorsten Brants, TnT: a statistical part-of-speech tagger, Proceedings of the sixth conference on Applied natural language processing, p.224-231, April 29-May 04, 2000, Seattle, Washington
\bibitem{StanfordTagger} Kristina Toutanova and Christopher D. Manning. 2000. Enriching the Knowledge Sources Used in a Maximum Entropy Part-of-Speech Tagger. In Proceedings of the Joint SIGDAT Conference on Empirical Methods in Natural Language Processing and Very Large Corpora (EMNLP/VLC-2000), pp. 63-70.
\bibitem{PennTreebank} Mitchell P. Marcus , Mary Ann Marcinkiewicz , Beatrice Santorini, Building a large annotated corpus of English: the penn treebank, Computational Linguistics, v.19 n.2, June 1993
\bibitem{BNC} Reference Guide for the British National Corpus (World Edition) edited by Lou Burnard, October 2000
%Definicion de corpus

\bibitem{TagMapping} Stevenson	M., A corpus-based approach to deriving lexical mappings,
 EACL '99 Proceedings of the ninth conference on European chapter of the Association for Computational Linguistics, 
Pages 285-286
\bibitem{DefCorpus1} Brown K. (Editor) 2005. Encyclopedia of Language and Linguistics � 2nd Edition. Oxford: Elsevier.
\bibitem{DefCorpus2} Sinclair, J. 'The automatic analysis of corpora', in Svartvik, J. (ed.) Directions in Corpus Linguistics (Proceedings of Nobel Symposium 82). Berlin: Mouton de Gruyter. 1992.
\end{thebibliography}
\section{An�lisis de error} 
Para mejorar el rendimiento de un etiquetador gramatical necesitamos entender donde est� funcionando mal. Por eso el an�lisis de error tiene un papel preponderante. Esta tarea se realiza construyendo una matriz de confusi�n o tabla de contingencia. Una matriz de confusi�n es una matriz de $n$ x $n$ donde la celda $(x,y)$ contiene el n�mero de veces que una palabra con correcta etiqueta $x$ fu� etiquetada por el modelo como $y$. Por ejemplo, la siguiente tabla muestra una porci�n de la matriz de confusi�n para los experimentos de etiquetado con HMM. 
\\
\\
\begin{center}
\begin{longtable}{| l | c | c | c | c | c | c | c |}
\caption{Ejemplo de matriz de confusi�n}\\	
\hline		
 &	\textbf{IN}	&	\textbf{JJ}	&	\textbf{NN}	&	\textbf{NNP} & \textbf{RB} & \textbf{VBD} & \textbf{VBN}\\
\hline
\endhead
\hline
\endfoot
\endlastfoot
	\hline
\textbf{IN}	&	-	&	.2	&	&	& .7 & & \\	
\textbf{JJ}	&	.2	&	-	&	3.3 & 2.1	& 1.7 & .2 & 2.7 \\	
\textbf{NN}	& &	8.7	&	- & & & & .2 \\	
\textbf{NNP} & .2 &	3.3	&	4.1 & - & .2 & & \\	
\textbf{RB} & 2.2 &	2.0	&	.5 &  & - & & \\	
\textbf{VBD} & & .3	&	.5 &  &  & - & 4.4\\	
\textbf{VBN} & & 2.8 & & & & 2.6 & -\\	
\hline
\end{longtable}
\end{center}
Las etiquetas de la fila indican las etiquetas correctas, las etiquetas de las columnas indican las etiquetas asignadas por el etiquetador, y cada celda indica el porcentaje del error de etiquetado general. Por lo tanto 4.4\% del total de errores fueron causados por fallida etiquetaci�n de VBD como VBN.
La matriz anterior y el an�lisis de error relacionado en \textsl{Franz, Kupiec y Ratnaparkhi} sugieren que algunos de los mayores problemas que encaran los etiquetadores actuales son:
\begin{enumerate}
	\item \textbf{NN contra NNP contra JJ:} Estas etiquetas son dif�ciles de distinguir. Es especialmente importante distinguir entre sustantivos propios para extracci�n de la informaci�n y traducci�n autom�tica.
	\item \textbf{RP contra RB contra IN:} Todas estas etiquetas pueden aparecer inmediatamente despu�s del verbo.
	\item \textbf{VBD contra VBN contra JJ:} Distinguir estas etiquetas es importante para el \textsl{parsing} parcial y para etiquetar correctamente los bordes de las frases nominales.
\end{enumerate}      
El an�lisis de error es una parte crucial de cualquier aplicaci�n ling��stica computacional. Puede ayudar a encontrar \textsl{bugs}, encontrar problemas en los datos de entrenamiento y lo m�s importante, ayuda en el desarrollo de conocimiento y/o algoritmos para utilizar en la soluci�n de problemas.
\section{Nuevo Corpus generado} 
A partir del corpus parcialmente anotado obtenido en el proceso de extracci�n, se completar�n las anotaciones autom�ticamente con un etiquetador gramatical manteniendo las etiquetas gramaticales obtenidas a partir de la informaci�n procedente del diccionario Cobuild. Es decir, una vez finalizado el proceso de extracci�n de informaci�n desde el diccionario, se obtiene un corpus nuevo con las etiquetas gramaticales correspondientes a las palabras definiadas en el diccionario. A continuaci�n se exhibe un fragmento del corpus extra�do de Cobuild con el formato generado:\\\\
{\small
\noindent A\\
canary	NN\\
is\\
a\\
small\\
yellow\\
bird\\
which\\
sings\\
beautifully\\
.\\
People\\
sometimes\\
keep\\
canaries	NNS\\
in\\
cages\\
as\\
pets\\
.\\
}

\noindent Este es el resultado de extracci�n, reconocimiento y traducci�n de etiquetas y formas flexionadas correspondiente a la entrada de Cobuild: 
{\small
\begin{verbatim}
DICTIONARY_ENTRY
k%en*!e*%eri
canary
canaries
A canary is a small yellow bird which sings beautifully. 
People sometimes keep canaries in cages as pets.  
countable noun
noun                                
\end{verbatim}
}
Se puede apreciar que se ha reconocido \textsl{canaries} como el plural de \textsl{canary} (etiqueta NNS) y que se han reconocido y extra�do los ejemplos de estas palabras asignando las etiquetas gramaticales traducidas a partir de las etiquetas del diccionario correspondientes a \textsl{canary} (countable noun/NN) y \textsl{canaries} (noun/NNS).

El pr�ximo paso ser� el de completar las anotaciones gramaticales para todas las palabras restantes. Este proceso se realiza anotando el corpus plano (sin las etiquetas obtenidas de Cobuild) con el etiquetador gramatical autom�tico TnT. Luego se une este corpus anotado por TnT con el corpus anotado parcialmente procedente de Cobuild, preservando todas las etiquetas del diccionario. 

El resultado que se muestra a continuaci�n es un nuevo corpus obtenido a partir de Cobuild, con las anotaciones que este provee y completado con anotaciones obtenidas mediante etiquetaci�n autom�tica utilizando TnT.\\\\
{\small
\begin{tabular}{l l}
A & DT\\
canary	& NN\\
is	& VBZ\\
a		& DT\\
small	& JJ\\
yellow	& JJ\\
bird	& NN\\
which	& WDT\\
sings	& VBZ\\
beautifully	& RB\\
.	& .\\
People	& NNS\\
sometimes	& RB\\
keep	& VB\\
canaries	& NNS\\
in	& IN\\
cages	& NNS\\
as	& IN\\
pets	& NNS\\
.	& .\\
\end{tabular}
}
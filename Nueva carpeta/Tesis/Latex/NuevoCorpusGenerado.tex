\section{Nuevo Corpus generado} 
A partir del corpus parcialmente anotado generado en la etapa anterior, se completar�n las anotaciones con un etiquetador autom�tico (TnT) preservando las etiquetas  obtenidas a partir de la informaci�n gramatical proveniente del diccionario \emph{Cobuild}. 

A continuaci�n se exhibe un ejemplo de este proceso:\\\\
\begin{samepage}
\noindent
Entrada \emph{Cobuild} para la palabra \emph{abide}
\begin{small}
\begin{verbatim}
DICTIONARY_ENTRY
abide
abides, abiding, abided
%eb*a*!id    
If something abides, it continues to happen or exist for a long time.
We feel the need to lean on something that abides.
verb
verb
\end{verbatim}
\end{small}
\end{samepage}
\\
\newpage
\begin{samepage}
\noindent 
Resultado de extracci�n, reconocimiento y traducci�n de etiquetas y formas flexionadas correspondiente a la entrada anterior: \\

\noindent 
\begin{small}
\emph{
\begin{tabular}{l l}
We&\\
feel&\\
the&\\
need&\\
to&\\
lean&\\
on&\\
something&\\
that&\\
abides	&VBZ\\
.&\\
\end{tabular}}
\end{small}
\end{samepage}

Se puede apreciar que en el ejemplo se ha reconocido \emph{abides} como verbo en tercera persona a partir de \emph{abide} y el r�tulo \emph{verb}, asignando la etiqueta gramatical traducida correspondiente: VBZ (obtenida por inferencia).

El pr�ximo paso ser� el de completar las anotaciones gramaticales para las palabras restantes. Este proceso se realiza anotando el corpus plano con el etiquetador gramatical autom�tico TnT, como puede verse en 2). Luego se une con el corpus anotado parcialmente procedente de Cobuild 1) con 2), preservando todas las etiquetas del diccionario. 

El resultado es un nuevo corpus obtenido a partir de Cobuild, con las anotaciones que este provee y completado con anotaciones autom�ticas 3).\\\\
\noindent 
\begin{small}
\begin{center}
\begin{frame}

\emph{
	\begin{tabular}{|l|l|}
	\hline
  \multicolumn{2}{|c|}{1) Ejemplo extra�do } \\
  \hline
	We&\\
feel&\\
the&\\
need&\\
to&\\
lean&\\
on&\\
something&\\
that&\\
abides&\textbf{VBZ}\\
.&\\
	\hline
	\end{tabular}
	$\longrightarrow$	
	\begin{tabular}{|l|l|}	
	\hline
  \multicolumn{2}{|c|}{2) Etiquetado autom�tico} \\
  \hline
	We			&PRP\\
	feel			&VBP\\
	the			&DT\\
	need			&NN\\
	to			&TO\\
	lean			&VB\\
	on			&IN\\
	something		&NN\\
	that			&IN\\
	abides&\textbf{NNS}\\
	.			&.\\
	\hline
	\end{tabular}
	$\longrightarrow$			
	\begin{tabular}{|l|l|}
	\hline
  \multicolumn{2}{|c|}{3) Nuevo corpus} \\
  \hline
	We			&PRP\\
	feel			&VBP\\
	the			&DT\\
	need			&NN\\
	to			&TO\\
	lean			&VB\\
	on			&IN\\
	something		&NN\\
	that			&IN\\
	abides			&\textbf{VBZ}\\
	.			&.\\
	\hline
	\end{tabular}}
\end{frame}

\end{center}
\end{small}

\subsection{Etiquetar el corpus WSJ con TnT} 
Este experimento consiste en entrenar TnT con WSJ y con WSJ + NFI. Luego se procede a etiquetar el WSJ plano (sin etiquetas gramaticales) con estos dos modelos. Por �ltimo se contruye la matriz de confusi�n:

\input{../Tesis/Bin/Debug/Datos/Experimentos/TnT/WSJ/Mediciones/wsj_VS_wsjTnTWsj.l}
\input{../Tesis/Bin/Debug/Datos/Experimentos/TnT/WSJ/Mediciones/wsj_VS_wsjTnTWsj+NFI.l}

Se puede observar que el rendimiento del etiquetador TnT entrenado con WSJ apenas mejor (97,39\%) que el rendimiento de TnT entrenado con WSJ+NFI (97,24\%). La mayor�a de los errores para TnT entrenado con WSJ se da en etiquetas NN del gold standard cuando son etiquetadas como JJ y NNP por TnT. Para TnT entrenado con WSJ + NFI la mayor�a de los errores se da en las mismas etiquetas, pero con cantidad de errores mayor, sobre todo para NN etiquetado como NNP.\\

La segunda evaluaci�n de este experimento consiste en entrenar TnT con la mitad de WSJ y con la mitad de WSJ + NFI. Posteriormente con estos dos modelos se etiqueta la mitad restante de WSJ y se construye la matriz de confusi�n. Se realiza la misma operaci�n para cada mitad:

\input{../Tesis/Bin/Debug/Datos/Experimentos/TnT/WSJ/Mediciones/wsjM1_VS_TnTWsjM2.l}
\input{../Tesis/Bin/Debug/Datos/Experimentos/TnT/WSJ/Mediciones/wsjM1_VS_TnTWsjM2+NFI.l}
\input{../Tesis/Bin/Debug/Datos/Experimentos/TnT/WSJ/Mediciones/wsjM2_VS_TnTWsjM1.l}
\input{../Tesis/Bin/Debug/Datos/Experimentos/TnT/WSJ/Mediciones/wsjM2_VS_TnTWsjM1+NFI.l}

Se puede apreciar una leve mejora en el porcentaje de etiquetas acertadas para el modelo que incorpora NFI; 96,25\% contra 96,47\% y 96,21\% contra 96,38\% para cada mitad respectivamente. Los errores m�s comunes son producidos en etiquetas NN del gold standard cuando son etiquetadas como JJ y NNP por TnT, para las dos mitades entrenadas tanto con WSJ como con WSJ + NFI. Se puede notar que el porcentaje de error al etiquetar JJ cuando era NN es menor en la evaluaci�n realizada sobre TnT entrenado con el modelo que incorpora NFI.\\


A continuaci�n se presenta la diferencia entre las etiquetas generadas a partir de WSJ vs WSJ + NFI. 

Espec�ficamente se muestran las matrices de confusi�n entre las mitades de WSJ etiquetado con TnT entrenado con la mitad restante con y sin NFI.

\input{../Tesis/Bin/Debug/Datos/Experimentos/TnT/WSJ/Mediciones/wsjM1__TnTWsjM2_vs_TnTWsjM2+NFI.l}
\input{../Tesis/Bin/Debug/Datos/Experimentos/TnT/WSJ/Mediciones/wsjM2__TnTWsjM1_vs_TnTWsjM1+NFI.l}	

La tercer evaluaci�n de este experimento consiste en entrenar TnT con un cuarto de WSJ y con un cuarto de WSJ + NFI. Posteriormente con estos dos modelos se etiqueta los 3/4 restantes de WSJ y se construye la matriz de confusi�n. Se realiza la misma operaci�n para cada uno de los cuartos.

En promedio de la cantidad de aciertos para el modelo entrenado con WSJ es 95.91\%, mientras que para el modelo entrenado con WSJ+NFI es 96.25\%

\begin{center}
\begin{table}[H]
\caption{Rendimiento de TnT entrenado con cuartos de WSJ con y sin NFI}\\	
\begin{tabular}{| l | c | }
\hline

 \textbf{Evaluaci�n}	&   \textbf{Porcentaje de aciertos}	&   \hline

\hline
TnT entrenado con el primer 1/4 de WSJ & 95.93\%  \\
TnT entrenado con el primer 1/4 de WSJ + NFI & 96.24\% \\
TnT entrenado con el segundo 1/4 de WSJ & 95.89\% \\
TnT entrenado con el segundo 1/4 de WSJ + NFI & 96.24\% \\
TnT entrenado con el tercer 1/4 de WSJ & 95.92\% \\
TnT entrenado con el tercer 1/4 de WSJ + NFI & 96.26\% \\
TnT entrenado con el cuarto 1/4 de WSJ & 95.90\% \\
TnT entrenado con el cuarto 1/4 de WSJ + NFI & 96.27\% \\
\hline
\end{tabular}
\end{table}
\end{center}

En todos los casos se puede apreciar una mejora en el acierto de etiquetas para el modelo que incorpora NFI.\\

La cuarta evaluaci�n de este experimento consiste en entrenar TnT con un d�cimo de WSJ y con un d�cimo de WSJ + NFI. Posteriormente con estos dos modelos se etiqueta los 9/10 restantes de WSJ y se presentan los resultados:
\begin{itemize}
	\item 95.32\% de acierto de etiquetas para el etiquetado de 9/10 de WSJ con TnT entrenado con 1/10 WSJ
	\item 96.02\% de acierto de etiquetas para el etiquetado de 9/10 de WSJ con TnT entrenado con 1/10 WSJ+NFI
\end{itemize}

Se puede apreciar un aumento del porcentaje de aciertos en el modelo que incorpora NFI.

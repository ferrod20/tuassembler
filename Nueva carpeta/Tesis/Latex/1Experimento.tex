\section{Primer experimento} 
El primer experimento consiste en medir (generando una matriz de confusi�n) la informaci�n extra�da de Cobuild contra la misma informaci�n generada a partir de un etiquetador autom�tico (TnT). De esta manera podremos observar la diferencia entre la informaci�n gramatical de Cobuild y la informaci�n que se podr�a generar autom�ticamente.

Como se mencion� anteriormente la informaci�n extra�da de Cobuild, es la uni�n de ejemplos con la informaci�n gramatical correspondiente a la palabra definida.
A continuaci�n se presenta un peque�o extracto:
\\\\
\noindent
\begin{small}
\emph{
\begin{tabular}{l l}
She\\
put\\
out\\
a\\
hand\\
and\\
stroked\\
the\\
\textbf{cat}	&\textbf{NN}\\
softly\\
...\\
\\
...\\
domestic\\
animals\\
such\\
as\\
dogs\\
and\\
\textbf{cats}	&\textbf{NNS}\\
.\\
\end{tabular}}
\end{small}
\\\\\\
Esta es la informaci�n extra�da de Cobuild para la palabra \emph{cat}; la uni�n de los ejemplos\\
\\
\begin{small}
\emph{She put out a hand and stroked the cat softly...}\\
\emph{...domestic animals such as dogs and cats.}\\
\end{small}
\\
Se puede notar la informaci�n gramatical expresada mediante las etiquetas NN y NNS para las palabras \emph{cat} y \emph{cats} respectivamente.
La idea de este experimento ser� comparar estas etiquetas contra las etiquetas asignadas por el etiquetador autom�tico TnT. Entonces se tomar� este corpus plano (sin etiquetas), se lo etiquetar� utilizando TnT entrenado con el corpus de entrenamiento WSJ y luego se realizar� la comparaci�n.\\
La matriz de confusi�n\footnote{Las matrices de confusi�n presentadas de aqu� en adelante contienen las primeras 10 etiquetas de mayor error} generada a partir de dicha comparaci�n es la siguiente:

\input{../Tesis/Bin/Debug/Datos/Extraccion/Cobuild2Pasada.l}


Se puede apreciar un alto porcentaje de aciertos entre las etiquetas extra�das de Cobuild (87,58\%) y las etiquetas asignadas por TnT. Este porcentaje indica que la informaci�n de etiquetas extra�das de Cobuild es consistente con las producidas por TnT. La mayor�a de los errores se da en etiquetas NN, JJ y VB de Cobuild cuando son etiquetadas como JJ, NN y VB y VBN por TnT respectivamente.
A continuaci�n se muestran algunas ejemplos de los errores:\\\\

\noindent Etiquetado por TnT como NN pero extra�do como VB de Cobuild
\begin{itemize}
	\item \textbf{share}: Lets share	the petrol costs...
	\item \textbf{name}: Name	the place, well be there...
\end{itemize}	
	
\noindent Etiquetado por TnT como JJ pero extra�do como NN de Cobuild
\begin{itemize}
	\item \textbf{flat}: A flat usually includes a kitchen and bathroom.
	\item \textbf{wireless}: messages sent by cable or wireless
\end{itemize}	
\noindent Etiquetado por TnT como NN pero extra�do como JJ de Cobuild
\begin{itemize}
	\item \textbf{firm}: Bake the cake for about an hour until it is firm and brown
	\item \textbf{kind}: I find them all very pleasant and extremely kind and helpful
\end{itemize}	

\noindent Etiquetado por TnT como VBN pero extra�do como JJ de Cobuild		
\begin{itemize}
	\item \textbf{settled}: They are practising settled agriculture
\end{itemize}	
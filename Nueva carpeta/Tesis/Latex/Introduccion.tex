\chapter{Introducci�n}

El etiquetado o anotado gramatical, tambi�n conocido como Part-of-speech tagging, POS tagging o simplemente POST, es el proceso de asignar una etiqueta gramatical a cada una de las palabras de un texto seg�n su categor�a l�xica. Por ejemplo tomemos la oraci�n siguiente:\\

\emph{There is no asbestos in our products now.} \\

\noindent El resultado de etiquetarla gramaticalmente es: \\

\noindent \emph{There/EX is/VBZ no/DT asbestos/NN in/IN our/PRP products/NNS now/RB ./.}\\

\noindent donde cada palabra est� sucedida por una barra oblicua seguida de la etiqueta gramatical asignada. Se puede apreciar por ejemplo que la palabra \textsl{is} fu� etiquetada como VBZ (verbo de tiempo presente en tercera persona singular), \textsl{products} fu� etiquetada como NNS (sustantivo plural), etc. Es decir que a cada palabra se le asign� un c�digo que se corresponde con una funci�n gramatical.

La complejidad del el etiquetado gramatical reside en la ambig�edad gramatical inherente al lenguaje. Por ejemplo, la palabra \textsl{premio} puede funcionar como sustantivo:\\

1) \emph{Gan� un premio} \\

\noindent o como verbo\\

2) \emph{Por tu esfuerzo te premio}\\

En 1), \emph{premio} tendr�a que recibir la etiqueta gramatical NN (sustantivo com�n) mientras que en 2) tendr�a que recibir la etiqueta gramatical VB (verbo). Las palabras circundantes a una palabra ambig�a como por ejemplo \emph{premio}, brindan informaci�n para deducir el sentido gramatical de la misma.

Como se menciona m�s adelante, el etiquetado gramatical juega un papel importante en �reas de la ling��stica computacional como s�ntesis del habla, reconocimiento del habla y recuperaci�n de la informaci�n. 

El etiquetado gramatical es realizado manualmente por ling�istas o autom�ticamente por programas conocidos como etiquetadores gramaticales. La mayor�a de las implementaciones actuales de estos programas est�n basadas en el aprendizaje; toman un corpus \footnote{Colecci�n de textos escritos y/o transcripciones del lenguaje oral para cierto idioma} anotado correctamente con el cual se entrenan y luego emplean el conocimiento adquirido para etiquetar el corpus objetivo. 

En esa primer etapa conocida como entrenamiento, el etiquetador gramatical obtiene y preserva informaci�n sobre cada palabra, su etiqueta asignada y su contexto. Posteriormente dado un corpus objetivo como entrada el etiquetador determina una etiqueta para cada palabra. Los resultados del etiquetado dependen en gran medida de la calidad, cantidad y representatividad sobre el dominio abordado de los datos de entrenamiento. 

Uno de los grandes problemas del etiquetado gramatical reside en la falta de corpus anotados para utilizar como datos de entrenamiento. Un corpus anotado es una tabla de palabras con su correspondiente etiqueta gramatical, como se muestra a continuaci�n:
\\\\ 
{\small\emph{
\begin{tabular}{l l}
Areas	&NNS\\
of	&IN\\
the	&DT\\
factory	&NN\\
were	&VBD\\
particularly	&RB\\
dusty	&JJ\\
where	&WRB\\
the	&DT\\
crocidolite	&NN\\
was	&VBD\\
used	&VBN\\
.	&.\\
\end{tabular}}}

La idea de este trabajo consiste en utilizar la informaci�n gramatical residente en los ejemplos de las entradas del diccionario Cobuild, etiquetar las palabras de los ejemplos que no poseen informaci�n gramatical con un etiquetador autom�tico y luego utilizar esta nueva fuente de informaci�n como corpus de entrenamiento. Por �ltimo analizar y medir el rendimiento de dos etiquetadores autom�ticos entrenados con esta fuente de informaci�n generada sobre dos corpora diferentes.

A continuaci�n se detalla m�s espec�ficamente cada una de las etapas que componen este trabajo. Identificar y comprender las entradas de Cobuild, extraer los ejemplos junto con su informaci�n gramatical, obtener etiquetas gramaticales complementarias para los ejemplos utilizando un etiquetasdor autom�tico. Utilizar esta nueva fuente de informaci�n como corpus de entrenamiento. Analizar los resultados utilizando distintos corpora y distintos etiqutadores.

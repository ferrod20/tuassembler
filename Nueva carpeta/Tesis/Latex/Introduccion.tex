\subsection{Motivaci�n}
El etiquetado o anotado gramatical, tambi�n conocido como Part-of-speech tagging, POS tagging o simplemente POST, es el proceso de asignar una etiqueta gramatical a cada una de las palabras de un texto seg�n su categor�a l�xica. Como se menciona m�s adelante, el etiquetado gramatical juega un papel importante en  �reas de la linguistica computacional como por ejemplo s�ntesis del habla, reconocimiento del habla y recuperaci�n de la informaci�n.

Este proceso es realizado manualmente por linguistas o autom�ticamente mediante programas conocidos como etiquetadores gramaticales. Estos programas son entrenados con el objetivo de que aprendan a etiquetar. Este entrenamiento se realiza utilizando un corpus anotado previamente denominado corpus de entrenamiento. De esta manera el etiquetador obtiene, procesa y retiene informaci�n sobre como est� anotado el corpus de entrenamiento que utiliza posteriormente durante el proceso de etiquetaci�n gramatical. 

Una vez finalizado el entrenamiento, se procede a etiquetar el texto requerido utilizando la informacion adquirida.

Basicamente un corpus etiquetado o anotado es una lista de palabras con su correspondiente etiqueta gramatical. Por ejemplo:
\\ \\
A			DT\\
form			NN\\
of			IN\\
asbestos		NN\\
once			RB\\
used			VBN\\
to			TO\\
make			VB\\
Kent			NNP\\
cigarette		NN\\
filters			NNS\\
has			VBZ\\
caused			VBN\\
a			DT\\
high			JJ\\
percentage		NN\\
of			IN\\
cancer			NN\\
deaths			NNS\\
among			IN\\
a			DT\\
group			NN\\
of			IN\\
workers			NNS\\
exposed			VBN\\
to			TO\\
it			PRP\\
more			RBR\\
than			IN\\
30			CD\\
years			NNS\\
ago			RB\\
,			,\\
researchers		NNS\\
reported		VBD\\
.			.\\ \\

La complejidad del etiquetado gramatical reside en que la etiqueta no depende solo de la palabra o tipo de palabra que se est� etiquetando, por el contrario, la ubicaci�n y el contexto de la palabra en donde �sta aparece es un factor determinante al asignar la etiqueta correspondiente.

Como fu� mencionado anteriormente, la etapa de entrenamiento de un etiquetador gramatical obtiene, procesa y retiene informaci�n sobre el contexto y la ubicaci�n de una palabra y la etiqueta asignada. Luego se utiliza esta informaci�n en el proceso de etiquetacion, donde el etiquetador analiza para cada palabra su ubicaci�n y contexto, y en base a ello y al conocimiento adquirido previamente con el corpus de entrenamiento determina una etiqueta gramatical.

Uno de los grandes problemas del etiquetado gramatical reside en la falta de corpus anotados para utilizar durante el entrenamiento.

Los corpus etiquetados (tambi�n llamados corpus de entrenamiento) son etiquetados manualmente por linguistas especializados. Es un trabajo profundamente meticuloso y tedioso ya que el linguista debe dar una etiqueta gramatical palabra por palabra, en corpus del orden del MILLON de palabras. Ademas de lo tedioso y complejo del trabajo, el tiempo empleado para etiquetar un corpus es sumamente extenso y como consecuencia el valor economico es realmente alto, ya que intervienen grupos de trabajo altamente capacitados durante un tiempo prolongado.

El resultado de este complejo proceso artesanal es una tabla de palabras con su correspondiente etiqueta gramatical como se mostr� anteriormente. Ante la importancia que adquieren los corpus etiquetados es inevitable pensar en alg�n otro tipo de texto que posea informaci�n de etiquetas. 

Por ejemplo un diccionario contiene una palabra, su definici�n y algunos ejemplos en donde �sta aparece con cada uno de sus sentidos. Es decir que de alguna manera un diccionario contiene por cada palabra uno o m�s contextos en donde �sta aparece etiquetada. Entonces si tomamos todos los ejemplos de cada palabra de un diccionario y su etiqueta podemos construir un corpus parcialmente anotado. 

\subsection{Trabajo realizado}
La idea de este trabajo es suplir la falta de corpus de enternamiento utilizando la informaci�n de etiquetado que posee un diccionario como una nueva fuente de informaci�n para entrenar etiquetadores autom�ticos. Utillizar esta informaci�n existente para etiquetar, medir y comparar resultados en el rendimiento conseguido. Combinar e integrar esta informaci�n obtenida con corpus de entrenamiento cl�sicos e inspeccionar resultados en el etiquetado.
Este trabajo menciona detalladamente la forma de extraer la informaci�n relevante de etiquetas gramaticales a partir de un diccionario y las decisiones que fueron aplicadas.
Por �ltimo se realizan mediciones entre esta nueva fuente de informaci�n y los corpus cl�sicos de entrenamiento y se presentan las conclusiones.



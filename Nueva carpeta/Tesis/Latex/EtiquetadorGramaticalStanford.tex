\section{Etiquetador Gramatical Stanford Tagger} 

Stanford Tagger es un etiquetador gramatical estoc�stico basado en el modelo de m�xima entrop�a.

Al igual que otros etiquetadores estoc�sticos, Stanford Tagger aprende de texto etiquetado: analiza y preserva informaci�n estad�stica sobre las palabras y las etiquetas asignadas. Dada una palabra $w$ y un contexto $h$\footnote{El contexto generalmente se define como una secuencia de varias palabras y etiquetas precediendo a la palabra actual} el modelo asigna una probabilidad a cada etiqueta $t$ perteneciente al conjunto de todas las etiquetas posibles $T$.

Etiquetado gramatical es el proceso de asignar la secuencia de etiquetas m�s probables a una cadena de palabras. La idea del modelo de m�xima entrop�a es elegir la distribuci�n de probabilidades p que tiene la mayor entrop�a entre todas las distribuciones que satisfacen ciertas restricciones.

Las restricciones obligan al modelo a comportarse de acuerdo a un conjunto de estad�sticas obtenidas del corpus de entrenamiento. Estas estad�sticas son expresadas como los valores esperados de funciones definidas sobre los conjuntos $h$ y las etiquetas $t$.

Por ejemplo si se quiere restringir el modelo a etiquetar la palabra \textsl{make} como verbo o sustantivo con la misma frecuencia del corpus de entrenamiento, se pueden definir las caracter�sticas:

\begin{equation*}f_1(h, t) = 1 \Leftrightarrow w_i = \text{make} \wedge t=\text{NN}\end{equation*}
\begin{equation*}f_2(h, t) = 1 \Leftrightarrow w_i = \text{make} \wedge t=\text{VB}\end{equation*}

A diferencia del modelo oculto de Markov, m�xima entrop�a permite definir e incorporar informaci�n estad�stica m�s compleja que informaci�n de frecuencia, bigramas y/o trigramas.

Stanford Tagger define caracter�sticas generales cl�sicas (bigramas, trigramas y frecuencia de etiquetas) y tambi�n caracter�sticas especiales para palabras raras, con el objetivo de mejorar la capacidad de predicci�n del modelo en palabras desconocidas.

\begin{itemize}
	\item \textbf{Caracter�sticas generales}
\begin{align*}
w_i = X \wedge t_i=T\\
t_{i-1} = T_1 \wedge t_i=T\\
t_{i-1} = T_1 \wedge t_{i-2} = T_2 \wedge t_i=T\\
w_{i+1} = X \wedge t_i=T\\
\end{align*}
 	\item \textbf{Caracter�sticas para palabras raras}
\begin{align*}
\text{El sufijo de } w_i = S \wedge |S|<5 \wedge t_i=T\\
\text{El prefijo de } w_i = P \wedge 1<|P|<5 \wedge t_i=T\\
w_i \text{ contiene un n�mero} \wedge t_i=T\\
w_i \text{ contiene una may�scula} \wedge t_i=T\\
w_i \text{ contiene un gui�n} \wedge t_i=T\\
\end{align*}
\end{itemize}

Las palabras raras son aquellas que aparecen pocas veces en el corpus de entrenamiento\footnote{Stanford Tagger toma como palabras raras aquellas que aparecen menos de 7 veces en el corpus}.

El rendimiento reportado para Stanford Tagger se encuentra dentro de los mismos par�metros de rendimiento de otros etiquetadores estoc�sticos. La ventaja es la capacidad de experimentar nuevas caracter�sticas o \textsl{features} que ayudan a mejorar su rendimiento.
\subsubsection{Etiquetar el corpus WSJ con Stanford Tagger} 
La segunda evaluaci�n de este experimento consiste en entrenar el etiquetador gramatical BNC con WSJ como corpus de entrenamiento y con WSJ + NFI. Luego se procede a etiquetar el WSJ plano (sin etiquetas gramaticales) con estos dos modelos. Por �ltimo se contruye la matriz de confusi�n:

\input{../Tesis/Bin/Debug/Datos/Experimentos/MaxEnt/WSJ/Mediciones/wsj_VS_wsjMaxEntWsj.l}
\input{../Tesis/Bin/Debug/Datos/Experimentos/MaxEnt/WSJ/Mediciones/wsj_VS_wsjMaxEntWsj+NFI.l}

Se puede observar que el rendimiento del etiquetador entrenado con WSJ es un poco mejor (97,9\%) que cuando es entrenado con WSJ + NFI (97,73\%). La mayor�a de los errores para Stanford Tagger entrenado con WSJ se da en etiquetas NN del gold standard cuando son etiquetadas como JJ y NNP. Para Stanford Tagger entrenado con WSJ + NFI la mayor�a de los errores se da en las mismas etiquetas, pero con cantidad de errores mayor, sobre todo para NN etiquetado como JJ.\\

La segunda evaluaci�n de este experimento consiste en entrenar Stanford Tagger con la mitad de WSJ y con la mitad de WSJ + NFI. Posteriormente con estos dos modelos se etiqueta la mitad restante de WSJ y se construye la matriz de confusi�n. Se realiza la misma operaci�n para cada mitad:

\input{../Tesis/Bin/Debug/Datos/Experimentos/MaxEnt/WSJ/Mediciones/wsjM1_VS_MaxEntWsjM2.l}
\input{../Tesis/Bin/Debug/Datos/Experimentos/MaxEnt/WSJ/Mediciones/wsjM1_VS_MaxEntWsjM2+NFI.l}
\input{../Tesis/Bin/Debug/Datos/Experimentos/MaxEnt/WSJ/Mediciones/wsjM2_VS_MaxEntWsjM1.l}
\input{../Tesis/Bin/Debug/Datos/Experimentos/MaxEnt/WSJ/Mediciones/wsjM2_VS_MaxEntWsjM1+NFI.l}

Se puede apreciar una leve mejor�a en el porcentaje de etiquetas acertadas; 96,23\% contra 96,46\% y 96,20\% contra 96,36\% para cada mitad respectivamente. Los errores m�s comunes son producidos en etiquetas NN del gold standard cuando son etiquetadas como JJ y NNP por TnT, para las dos mitades entrenadas tanto con WSJ como con WSJ + NFI. Se puede notar que el porcentaje de error al etiquetar JJ cuando era NN es menor en la evaluaci�n realizada sobre TnT entrenado con WSJ + NFI.\\

A continuaci�n se presentan las matrices de confusi�n entre las mitades de WSJ etiquetado con Stanford Tagger entrenado con la mitad restante con y sin NFI.

\input{../Tesis/Bin/Debug/Datos/Experimentos/MaxEnt/WSJ/Mediciones/wsjM1__MaxEntWsjM2_vs_MaxEntWsjM2+NFI.l}
\input{../Tesis/Bin/Debug/Datos/Experimentos/MaxEnt/WSJ/Mediciones/wsjM2__MaxEntWsjM1_vs_MaxEntWsjM1+NFI.l}	

La tercer evaluaci�n de este experimento consiste en entrenar Stanford Tagger con un cuarto de WSJ y con un cuarto de WSJ + NFI. Posteriormente con estos dos modelos se etiqueta los 3/4 restantes de WSJ y se construye la matriz de confusi�n. Se realiza la misma operaci�n para cada uno de los cuartos:

\begin{center}
\begin{longtable}{| l | c | }
\caption{Rendimiento de TnT entrenado con cuartos de WSJ con y sin NFI}\\	
\hline
 \textbf{Evaluaci�n}	&   \textbf{Porcentaje de aciertos}	&   \hline
\endhead
\hline
\endfoot
\endlastfoot
	\hline
Stanford Tagger entrenado con el primer 1/4 de WSJ & 96.30\%  \\
Stanford Tagger entrenado con el primer 1/4 de WSJ + NFI & 96.57\% \\
Stanford Tagger entrenado con el segundo 1/4 de WSJ & 96.30\% \\
Stanford Tagger entrenado con el segundo 1/4 de WSJ + NFI & 96.52\% \\
Stanford Tagger entrenado con el tercer 1/4 de WSJ & 96.28\% \\
Stanford Tagger entrenado con el tercer 1/4 de WSJ + NFI & 96.57\% \\
Stanford Tagger entrenado con el cuarto 1/4 de WSJ & 96.24\% \\
Stanford Tagger entrenado con el cuarto 1/4 de WSJ + NFI & 96.53\% \\
\hline
\end{longtable}
\end{center}

En todos los casos se puede apreciar una mejora en el acierto de etiquetas para el corpus de entrenamiento WSJ + NFI contra WSJ.\\

La cuarta evaluaci�n de este experimento consiste en entrenar Stanford Tagger con un d�cimo de WSJ y con un d�cimo de WSJ + NFI. Posteriormente con estos dos modelos se etiqueta los 9/10 restantes de WSJ y se presentan los resultados:
\begin{itemize}
	\item 95.67\% de acierto de etiquetas para el etiquetado de 9/10 de WSJ con Stanford Tagger entrenado con 1/10 WSJ
	\item 96.27\% de acierto de etiquetas para el etiquetado de 9/10 de WSJ con Stanford Tagger entrenado con 1/10 WSJ+NFI
\end{itemize}

Se puede apreciar un aumento del porcentaje de aciertos en el corpus de entrenamiento que incorpora NFI.

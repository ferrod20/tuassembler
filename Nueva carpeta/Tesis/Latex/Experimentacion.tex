\chapter{Experimentaci�n} 
\section{Primer experimento} 
El primer experimento consiste en medir (generando una matriz de confusi�n) la informaci�n extra�da de Cobuild contra la misma informaci�n generada a partir de un etiquetador autom�tico (TnT). De esta manera podremos observar la diferencia entre la informaci�n gramatical de Cobuild y la informaci�n que se podr�a generar autom�ticamente.

Como se mencion� anteriormente la informaci�n extra�da de Cobuild, es la uni�n de ejemplos con la informaci�n gramatical correspondiente a la palabra definida.
A continuaci�n se presenta un peque�o extracto:\\\\
\noindent{\small
\begin{tabular}{l l}
Cats	&NNS \\
are\\
often\\
kept\\
as\\
pets\\
.\\
She\\
put\\
out\\
a\\
hand\\
and\\
stroked\\
the\\
cat	&NN\\
softly\\
...\\
...\\
domestic\\
animals\\
such\\
as\\
dogs\\
and\\
cats	&NNS\\
.\\
\end{tabular}}
\\\\
Esta es la informaci�n extra�da de Cobuild para la palabra \emph{cat}; la uni�n de los ejemplos\\
\\
\emph{Cats are often kept as pets.}\\
\emph{She put out a hand and stroked the cat softly...}\\
\emph{...domestic animals such as dogs and cats.}\\
\\
Se puede notar la informaci�n gramatical expresada mediante las etiquetas NN y NNS para las palabras \emph{cat} y \emph{cats} respectivamente.
La idea de este experimento ser� comparar estas etiquetas contra las etiquetas asignadas por el etiquetador autom�tico TnT. Entonces se tomar� este corpus plano (sin etiquetas), se lo etiquetar� utilizando TnT entrenado con el corpus de entrenamiento Wall Street Journal (de ahora en m�s WSJ) \footnote{Wall Street Journal es un corpus anotado, parte del Penn Treebank} y luego se realizar� la comparaci�n.\\
La matriz de confusi�n\footnote{Las matrices de confusi�n presentadas de aqu� en adelante contienen las primeras 10 etiquetas de mayor error} generada a partir de dicha comparaci�n es la siguiente:

\input{../Tesis/Bin/Debug/Datos/Extraccion/Cobuild2Pasada.l}


Se puede apreciar un alto porcentaje de aciertos entre las etiquetas extra�das de Cobuild (86,75\%) y las etiquetas asignadas por TnT. Este porcentaje indica que la informaci�n de etiquetas extra�das de Cobuild es consistente con las producidas por TnT. La mayor�a de los errores se da en etiquetas VB, NN y JJ de Cobuild cuando son etiquetadas como NN, JJ y VBN por TnT respectivamente.
A continuaci�n se muestran algunas ejemplos de los errores:\\\\

\noindent Etiquetado por TnT como NN pero extra�do como VB de Cobuild
\begin{itemize}
	\item \textbf{share}: Lets share	the petrol costs...
	\item \textbf{name}: Name	the place, well be there...
\end{itemize}	
	
\noindent Etiquetado por TnT como JJ pero extra�do como NN de Cobuild
\begin{itemize}
	\item \textbf{flat}: A flat usually includes a kitchen and bathroom.
	\item \textbf{wireless}: messages sent by cable or wireless
\end{itemize}	
\noindent Etiquetado por TnT como NN pero extra�do como JJ de Cobuild
\begin{itemize}
	\item \textbf{firm}: Bake the cake for about an hour until it is firm and brown
	\item \textbf{kind}: I find them all very pleasant and extremely kind and helpful
\end{itemize}	

\noindent Etiquetado por TnT como VBN pero extra�do como JJ de Cobuild		
\begin{itemize}
	\item \textbf{settled}: They are practising settled agriculture
\end{itemize}	 

\newpage
\section{Segundo experimento: Etiquetar el corpus WSJ} 
El segundo experimento realizado tiene como objetivo evaluar la nueva fuente de informaci�n obtenida (NFI) como corpus de entrenamiento. Para esto se entrenar�n 2 etiquetadores gramaticales (Stanford Tagger y TnT) y se etiquetar� con ellos el corpus Wall Street Journal (WSJ). Posteriormente se realizar�n mediciones de desempe�o pertinentes.
\subsubsection{Etiquetar el corpus WSJ con TnT} 
La primer evaluaci�n de este segundo experimento consiste en entrenar el etiquetador gramatical TnT con WSJ como corpus de entrenamiento y con WSJ + NFI. Luego se procede a etiquetar el WSJ plano (sin etiquetas gramaticales) con estos dos modelos. Por �ltimo se contruye la matriz de confusi�n:

\input{../Tesis/Bin/Debug/Datos/Experimentos/TnT/WSJ/Mediciones/wsj_VS_wsjTnTWsj.l}
\input{../Tesis/Bin/Debug/Datos/Experimentos/TnT/WSJ/Mediciones/wsj_VS_wsjTnTWsj+NFI.l}

Se puede observar que el rendimiento del etiquetador TnT entrenado con WSJ+NFI es un poco mejor (97,14\%) que el rendimiento de TnT entrenado con WSJ (97,1\%). La mayor�a de los errores para TnT entrenado con WSJ se da en etiquetas NN del gold standard cuando son etiquetadas como JJ y NNP por TnT. Para TnT entrenado con WSJ + NFI la mayor�a de los errores se da en las mismas etiquetas, pero con cantidad de errores mayor, sobre todo para NN etiquetado como NNP.\\

La segunda evaluaci�n de este experimento consiste en entrenar TnT con la mitad de WSJ y con la mitad de WSJ + NFI. Posteriormente con estos dos modelos se etiqueta la mitad restante de WSJ y se construye la matriz de confusi�n. Se realiza la misma operaci�n para cada mitad:

\input{../Tesis/Bin/Debug/Datos/Experimentos/TnT/WSJ/Mediciones/wsjM1_VS_TnTWsjM2.l}
\input{../Tesis/Bin/Debug/Datos/Experimentos/TnT/WSJ/Mediciones/wsjM1_VS_TnTWsjM2+NFI.l}
\input{../Tesis/Bin/Debug/Datos/Experimentos/TnT/WSJ/Mediciones/wsjM2_VS_TnTWsjM1.l}
\input{../Tesis/Bin/Debug/Datos/Experimentos/TnT/WSJ/Mediciones/wsjM2_VS_TnTWsjM1+NFI.l}

Se puede apreciar una leve mejor�a en el porcentaje de etiquetas acertadas para el modelo que incorpora NFI; 96,25\% contra 96,47\% y 96,21\% contra 96,37\% para cada mitad respectivamente. Los errores m�s comunes son producidos en etiquetas NN del gold standard cuando son etiquetadas como JJ y NNP por TnT, para las dos mitades entrenadas tanto con WSJ como con WSJ + NFI. Se puede notar que el porcentaje de error al etiquetar JJ cuando era NN es menor en la evaluaci�n realizada sobre TnT entrenado con el modelo que incorpora NFI.\\

A continuaci�n se presentan las matrices de confusi�n entre las mitades de WSJ etiquetado con TnT entrenado con la mitad restante con y sin NFI.

\input{../Tesis/Bin/Debug/Datos/Experimentos/TnT/WSJ/Mediciones/wsjM1__TnTWsjM2_vs_TnTWsjM2+NFI.l}
\input{../Tesis/Bin/Debug/Datos/Experimentos/TnT/WSJ/Mediciones/wsjM2__TnTWsjM1_vs_TnTWsjM1+NFI.l}	

La tercer evaluaci�n de este experimento consiste en entrenar TnT con un cuarto de WSJ y con un cuarto de WSJ + NFI. Posteriormente con estos dos modelos se etiqueta los 3/4 restantes de WSJ y se construye la matriz de confusi�n. Se realiza la misma operaci�n para cada uno de los cuartos:

\begin{center}
\begin{longtable}{| l | c | }
\caption{Rendimiento de TnT entrenado con cuartos de WSJ con y sin NFI}\\	
\hline
 \textbf{Evaluaci�n}	&   \textbf{Porcentaje de aciertos}	&   \hline
\endhead
\hline
\endfoot
\endlastfoot
	\hline
TnT entrenado con el primer 1/4 de WSJ & 95.93\%  \\
TnT entrenado con el primer 1/4 de WSJ + NFI & 96.26\% \\
TnT entrenado con el segundo 1/4 de WSJ & 95.89\% \\
TnT entrenado con el segundo 1/4 de WSJ + NFI & 96.26\% \\
TnT entrenado con el tercer 1/4 de WSJ & 95.91\% \\
TnT entrenado con el tercer 1/4 de WSJ + NFI & 96.29\% \\
TnT entrenado con el cuarto 1/4 de WSJ & 95.9\% \\
TnT entrenado con el cuarto 1/4 de WSJ + NFI & 96.30\% \\
\hline
\end{longtable}
\end{center}

En todos los casos se puede apreciar una mejora en el acierto de etiquetas para el modelo que incorpora NFI.\\

La cuarta evaluaci�n de este experimento consiste en entrenar TnT con un d�cimo de WSJ y con un d�cimo de WSJ + NFI. Posteriormente con estos dos modelos se etiqueta los 9/10 restantes de WSJ y se presentan los resultados:
\begin{itemize}
	\item 95.32\% de acierto de etiquetas para el etiquetado de 9/10 de WSJ con TnT entrenado con 1/10 WSJ
	\item 96.1\% de acierto de etiquetas para el etiquetado de 9/10 de WSJ con TnT entrenado con 1/10 WSJ+NFI
\end{itemize}

Se puede apreciar un aumento del porcentaje de aciertos en el modelo que incorpora NFI.
 
\newpage
\subsubsection{Etiquetar el corpus WSJ con Stanford Tagger} 
La segunda evaluaci�n de este experimento consiste en entrenar el etiquetador gramatical Stanford Tagger con WSJ como corpus de entrenamiento y con WSJ + NFI. Luego se procede a etiquetar el WSJ plano (sin etiquetas gramaticales) con estos dos modelos. Por �ltimo se contruye la matriz de confusi�n:

\input{../Tesis/Bin/Debug/Datos/Experimentos/MaxEnt/WSJ/Mediciones/wsj_VS_wsjMaxEntWsj.l}
\input{../Tesis/Bin/Debug/Datos/Experimentos/MaxEnt/WSJ/Mediciones/wsj_VS_wsjMaxEntWsj+NFI.l}

Se puede observar que el rendimiento del etiquetador entrenado con WSJ es un poco mejor (97,9\%) que cuando es entrenado con WSJ + NFI (97,73\%). La mayor�a de los errores para Stanford Tagger entrenado con WSJ se da en etiquetas NN del gold standard cuando son etiquetadas como JJ y NNP. Para Stanford Tagger entrenado con WSJ + NFI la mayor�a de los errores se da en las mismas etiquetas, pero con cantidad de errores mayor, sobre todo para NN etiquetado como JJ.\\

La segunda evaluaci�n de este experimento consiste en entrenar Stanford Tagger con la mitad de WSJ y con la mitad de WSJ + NFI. Posteriormente con estos dos modelos se etiqueta la mitad restante de WSJ y se construye la matriz de confusi�n. Se realiza la misma operaci�n para cada mitad:

\input{../Tesis/Bin/Debug/Datos/Experimentos/MaxEnt/WSJ/Mediciones/wsjM1_VS_MaxEntWsjM2.l}
\input{../Tesis/Bin/Debug/Datos/Experimentos/MaxEnt/WSJ/Mediciones/wsjM1_VS_MaxEntWsjM2+NFI.l}
\input{../Tesis/Bin/Debug/Datos/Experimentos/MaxEnt/WSJ/Mediciones/wsjM2_VS_MaxEntWsjM1.l}
\input{../Tesis/Bin/Debug/Datos/Experimentos/MaxEnt/WSJ/Mediciones/wsjM2_VS_MaxEntWsjM1+NFI.l}

Se puede apreciar una leve mejor�a en el porcentaje de etiquetas acertadas; 96,23\% contra 96,46\% y 96,20\% contra 96,36\% para cada mitad respectivamente. Los errores m�s comunes son producidos en etiquetas NN del gold standard cuando son etiquetadas como JJ y NNP por TnT, para las dos mitades entrenadas tanto con WSJ como con WSJ + NFI. Se puede notar que el porcentaje de error al etiquetar JJ cuando era NN es menor en la evaluaci�n realizada sobre TnT entrenado con WSJ + NFI.\\

A continuaci�n se presentan las matrices de confusi�n entre las mitades de WSJ etiquetado con Stanford Tagger entrenado con la mitad restante con y sin NFI.

\input{../Tesis/Bin/Debug/Datos/Experimentos/MaxEnt/WSJ/Mediciones/wsjM1__MaxEntWsjM2_vs_MaxEntWsjM2+NFI.l}
\input{../Tesis/Bin/Debug/Datos/Experimentos/MaxEnt/WSJ/Mediciones/wsjM2__MaxEntWsjM1_vs_MaxEntWsjM1+NFI.l}	

La tercer evaluaci�n de este experimento consiste en entrenar Stanford Tagger con un cuarto de WSJ y con un cuarto de WSJ + NFI. Posteriormente con estos dos modelos se etiqueta los 3/4 restantes de WSJ y se construye la matriz de confusi�n. Se realiza la misma operaci�n para cada uno de los cuartos:

\begin{center}
\begin{longtable}{| l | c | }
\caption{Rendimiento de Stanford Tagger entrenado con cuartos de WSJ con y sin NFI}\\	
\hline
 \textbf{Evaluaci�n}	&   \textbf{Porcentaje de aciertos}	&   \hline
\endhead
\hline
\endfoot
\endlastfoot
	\hline
Stanford Tagger entrenado con el primer 1/4 de WSJ & 96.30\%  \\
Stanford Tagger entrenado con el primer 1/4 de WSJ + NFI & 96.57\% \\
Stanford Tagger entrenado con el segundo 1/4 de WSJ & 96.30\% \\
Stanford Tagger entrenado con el segundo 1/4 de WSJ + NFI & 96.52\% \\
Stanford Tagger entrenado con el tercer 1/4 de WSJ & 96.28\% \\
Stanford Tagger entrenado con el tercer 1/4 de WSJ + NFI & 96.57\% \\
Stanford Tagger entrenado con el cuarto 1/4 de WSJ & 96.24\% \\
Stanford Tagger entrenado con el cuarto 1/4 de WSJ + NFI & 96.53\% \\
\hline
\end{longtable}
\end{center}

En todos los casos se puede apreciar una mejora en el acierto de etiquetas para el corpus de entrenamiento WSJ + NFI contra WSJ.\\

La cuarta evaluaci�n de este experimento consiste en entrenar Stanford Tagger con un d�cimo de WSJ y con un d�cimo de WSJ + NFI. Posteriormente con estos dos modelos se etiqueta los 9/10 restantes de WSJ y se presentan los resultados:
\begin{itemize}
	\item 95.67\% de acierto de etiquetas para el etiquetado de 9/10 de WSJ con Stanford Tagger entrenado con 1/10 WSJ
	\item 96.27\% de acierto de etiquetas para el etiquetado de 9/10 de WSJ con Stanford Tagger entrenado con 1/10 WSJ+NFI
\end{itemize}

Se puede apreciar un aumento del porcentaje de aciertos en el corpus de entrenamiento que incorpora NFI.
 
\newpage

\section{Tercer experimento: Etiquetar el corpus BNC} 
El tercer experimento realizado tiene como objetivo evaluar la nueva fuente de informaci�n obtenida (NFI) como corpus de entrenamiento. Para esto se entrenar�n 2 etiquetadores gramaticales (Stanford Tagger y TnT) y se etiquetar� con ellos el British National Corpus (BNC).
\subsubsection{Etiquetar el corpus BNC con TnT} 
La primer evaluaci�n de este experimento consiste en entrenar el etiquetador gramatical TnT con WSJ como corpus de entrenamiento y con WSJ + NFI. Luego se procede a etiquetar el BNC plano (sin etiquetas gramaticales) con estos dos modelos. Por �ltimo se contruye la matriz de confusi�n:

\input{../Tesis/Bin/Debug/Datos/Experimentos/TnT/BNC/Mediciones/bnc_VS_bncTnTWSJ.l}
\input{../Tesis/Bin/Debug/Datos/Experimentos/TnT/BNC/Mediciones/bnc_VS_bncTnTWsj+NFI.l}

Se puede observar que el rendimiento del etiquetador TnT entrenado con WSJ+NFI es un poco mejor (97,14\%) que el rendimiento de TnT entrenado con WSJ (97,1\%). La mayor�a de los errores para TnT entrenado con WSJ se da en etiquetas NN del gold standard cuando son etiquetadas como JJ y NNP por TnT. Para TnT entrenado con WSJ + NFI la mayor�a de los errores se da en las mismas etiquetas, pero con cantidad de errores mayor, sobre todo para NN etiquetado como NNP.\\

La segunda evaluaci�n de este experimento consiste en entrenar TnT con la mitad de WSJ y con la mitad de WSJ + NFI. Posteriormente con estos dos modelos se etiqueta BNC y se construye la matriz de confusi�n. Se realiza la misma operaci�n para cada mitad:

\input{../Tesis/Bin/Debug/Datos/Experimentos/TnT/BNC/Mediciones/bnc_VS_bncTnTWsjM2.l}
\input{../Tesis/Bin/Debug/Datos/Experimentos/TnT/BNC/Mediciones/bnc_VS_bncTnTWsjM2+NFI.l}
\input{../Tesis/Bin/Debug/Datos/Experimentos/TnT/BNC/Mediciones/bnc_VS_bncTnTWsjM1.l}
\input{../Tesis/Bin/Debug/Datos/Experimentos/TnT/BNC/Mediciones/bnc_VS_bncTnTWsjM1+NFI.l}

Se puede apreciar una leve mejor�a en el porcentaje de etiquetas acertadas para el modelo que incorpora NFI; 92,09\% contra 92,67\% y 92,14\% contra 92,7\% para cada mitad respectivamente. Los errores m�s comunes son producidos en etiquetas NN del gold standard cuando son etiquetadas como JJ y NNP por TnT, para las dos mitades entrenadas tanto con WSJ como con WSJ + NFI. Se puede notar que el porcentaje de error al etiquetar JJ cuando era NN es menor en la evaluaci�n realizada sobre TnT entrenado con el modelo que incorpora NFI.\\

A continuaci�n se presentan las matrices de confusi�n entre las mitades de WSJ etiquetado con TnT entrenado con la mitad restante con y sin NFI.

\input{../Tesis/Bin/Debug/Datos/Experimentos/TnT/BNC/Mediciones/bncTnTwsjM1_vs_bncTnTwsjM1+NFI.l}
\input{../Tesis/Bin/Debug/Datos/Experimentos/TnT/BNC/Mediciones/bncTnTwsjM2_vs_bncTnTwsjM2+NFI.l}	

La tercer evaluaci�n de este experimento consiste en entrenar TnT con un cuarto de WSJ y con un cuarto de WSJ + NFI. Posteriormente con estos dos modelos se etiqueta los 3/4 restantes de WSJ y se construye la matriz de confusi�n. Se realiza la misma operaci�n para cada uno de los cuartos:

\begin{center}
\begin{longtable}{| l | c | }
\caption{Rendimiento de TnT entrenado con cuartos de WSJ con y sin NFI}\\	
\hline
 \textbf{Evaluaci�n}	&   \textbf{Porcentaje de aciertos}	&   \hline
\endhead
\hline
\endfoot
\endlastfoot
	\hline
TnT entrenado con el primer 1/4 de WSJ & 91.75\%  \\
TnT entrenado con el primer 1/4 de WSJ + NFI & 92.62\% \\
TnT entrenado con el segundo 1/4 de WSJ & 91.74\% \\
TnT entrenado con el segundo 1/4 de WSJ + NFI & 92.63\% \\
TnT entrenado con el tercer 1/4 de WSJ & 91.64\% \\
TnT entrenado con el tercer 1/4 de WSJ + NFI & 92.62\% \\
TnT entrenado con el cuarto 1/4 de WSJ & 91.64\% \\
TnT entrenado con el cuarto 1/4 de WSJ + NFI & 92.58\% \\
\hline
\end{longtable}
\end{center}

En todos los casos se puede apreciar una mejora en el acierto de etiquetas para el modelo que incorpora NFI.\\

La cuarta evaluaci�n de este experimento consiste en entrenar TnT con un d�cimo de WSJ y con un d�cimo de WSJ + NFI. Posteriormente con estos dos modelos se etiqueta BNC y se presentan los resultados:
\begin{itemize}
	\item 90.9\% de acierto de etiquetas para el etiquetado de BNC con TnT entrenado con 1/10 WSJ
	\item 92.55\% de acierto de etiquetas para el etiquetado de BNC con TnT entrenado con 1/10 WSJ+NFI
\end{itemize}

Se puede apreciar un aumento del porcentaje de aciertos en el modelo que incorpora NFI.
 
\newpage
\subsubsection{Etiquetar el corpus BNC con Stanford Tagger} 
La segunda evaluaci�n de este experimento consiste en entrenar el etiquetador gramatical Stanford Tagger con WSJ como corpus de entrenamiento y con WSJ + NFI. Luego se procede a etiquetar el BNC plano (sin etiquetas gramaticales) con estos dos modelos. Por �ltimo se contruye la matriz de confusi�n:

\input{../Tesis/Bin/Debug/Datos/Experimentos/MaxEnt/BNC/Mediciones/bnc_VS_bncMaxEntWSJ.l}
\input{../Tesis/Bin/Debug/Datos/Experimentos/MaxEnt/BNC/Mediciones/bnc_VS_bncMaxEntWSJ+NFI.l}

Se puede observar que el rendimiento del etiquetador entrenado con WSJ es un poco mejor (93,01\%) que cuando es entrenado con WSJ + NFI (92,86\%). La mayor�a de los errores para Stanford Tagger entrenado con WSJ se da en etiquetas NN del gold standard cuando son etiquetadas como JJ y NNP. Para Stanford Tagger entrenado con WSJ + NFI la mayor�a de los errores se da en las mismas etiquetas, pero con cantidad de errores mayor, sobre todo para NN etiquetado como JJ.\\

La segunda evaluaci�n de este experimento consiste en entrenar Stanford Tagger con la mitad de WSJ y con la mitad de WSJ + NFI. Posteriormente con estos dos modelos se etiqueta BNC y se construye la matriz de confusi�n. Se realiza la misma operaci�n para cada mitad:

\input{../Tesis/Bin/Debug/Datos/Experimentos/MaxEnt/BNC/Mediciones/bnc_VS_bncMaxEntWSJM2.l}
\input{../Tesis/Bin/Debug/Datos/Experimentos/MaxEnt/BNC/Mediciones/bnc_VS_bncMaxEntWSJM2+NFI.l}
\input{../Tesis/Bin/Debug/Datos/Experimentos/MaxEnt/BNC/Mediciones/bnc_VS_bncMaxEntWSJM1.l}
\input{../Tesis/Bin/Debug/Datos/Experimentos/MaxEnt/BNC/Mediciones/bnc_VS_bncMaxEntWSJM1+NFI.l}

Se puede apreciar una leve mejor�a en el porcentaje de etiquetas acertadas; 92,6\% contra 93,01\% y 92,45\% contra 92,91\% para cada modelo respectivamente. Los errores m�s comunes son producidos en etiquetas NN del gold standard cuando son etiquetadas como JJ y NNP por , para las dos mitades entrenadas tanto con WSJ como con WSJ + NFI. Se puede notar que el porcentaje de error al etiquetar JJ cuando era NN es menor en la evaluaci�n realizada sobre Stanford Tagger entrenado con WSJ + NFI.\\

A continuaci�n se presentan las matrices de confusi�n para BNC etiquetado con Stanford Tagger entrenado con la mitad de WSJ con y sin NFI.

\input{../Tesis/Bin/Debug/Datos/Experimentos/MaxEnt/BNC/Mediciones/bncMaxEntWSJM2_vs_bncMaxEntWSJM2+NFI.l}
\input{../Tesis/Bin/Debug/Datos/Experimentos/MaxEnt/BNC/Mediciones/bncMaxEntWSJM1_vs_bncMaxEntWSJM1+NFI.l}	

La tercer evaluaci�n de este experimento consiste en entrenar Stanford Tagger con un cuarto de WSJ y con un cuarto de WSJ + NFI. Posteriormente con estos dos modelos se etiqueta BNC y se construye la matriz de confusi�n. Se realiza la misma operaci�n para cada uno de los cuartos:

\begin{center}
\begin{longtable}{| l | c | }
\caption{Rendimiento de Stanford Tagger entrenado con cuartos de WSJ con y sin NFI}\\	
\hline
 \textbf{Evaluaci�n}	&   \textbf{Porcentaje de aciertos}	&   \hline
\endhead
\hline
\endfoot
\endlastfoot
	\hline
Stanford Tagger entrenado con el primer 1/4 de WSJ & 92.09\%  \\
Stanford Tagger entrenado con el primer 1/4 de WSJ + NFI & 92.92\% \\
Stanford Tagger entrenado con el segundo 1/4 de WSJ & 92.10\% \\
Stanford Tagger entrenado con el segundo 1/4 de WSJ + NFI & 92.91\% \\
Stanford Tagger entrenado con el tercer 1/4 de WSJ & 92.14\% \\
Stanford Tagger entrenado con el tercer 1/4 de WSJ + NFI & 92.89\% \\
Stanford Tagger entrenado con el cuarto 1/4 de WSJ & 91.98\% \\
Stanford Tagger entrenado con el cuarto 1/4 de WSJ + NFI & 92.83\% \\
\hline
\end{longtable}
\end{center}

En todos los casos se puede apreciar una mejora en el acierto de etiquetas para el corpus de entrenamiento WSJ + NFI contra WSJ.\\

La cuarta evaluaci�n de este experimento consiste en entrenar Stanford Tagger con un d�cimo de WSJ y con un d�cimo de WSJ + NFI. Posteriormente con estos dos modelos se etiqueta BNC y se presentan los resultados:
\begin{itemize}
	\item 91.25\% de acierto de etiquetas para el etiquetado de BNC con Stanford Tagger entrenado con 1/10 WSJ
	\item 92.81\% de acierto de etiquetas para el etiquetado de BNC con Stanford Tagger entrenado con 1/10 WSJ+NFI
\end{itemize}

Se puede apreciar un aumento del porcentaje de aciertos en el corpus de entrenamiento que incorpora NFI.
 
\newpage

\chapter{Conclusiones} 
Utilizar un diccionario como nueva fuente de informaci�n, convirti�ndolo en un corpus de entrenamiento para etiquetadores gramaticales aumenta levemente el rendimiento final del etiquetado. Esto es cierto incluso para etiquetadores de distintas bases te�ricas (m�xima entrop�a y modelos ocultos de Markov).
Las mejoras no logran ser significativas y aumentan t�midamente los valores del resultado final. 

Esto puede suceder ya que la cantidad de informaci�n gramatical que agrega un diccionario no es tan considerable; asciende a un valor cercano al 8\% de etiquetas por palabra, es decir que de cada 100 palabras que se extraen de los ejemplos del diccionario solo 8 poseen una etiqueta gramatical.


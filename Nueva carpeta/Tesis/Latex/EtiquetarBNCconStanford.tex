\subsubsection{Etiquetar el corpus BNC con Stanford Tagger} 
La primer evaluaci�n de este experimento consiste en entrenar el etiquetador gramatical Stanford Tagger con WSJ como corpus de entrenamiento y con WSJ + NFI. Luego se procede a etiquetar el BNC plano (sin etiquetas gramaticales) con estos dos modelos. Por �ltimo se contruye la matriz de confusi�n:

La segunda evaluaci�n de este experimento consiste en entrenar Stanford Tagger con la mitad de WSJ y con la mitad de WSJ + NFI. Posteriormente con estos dos modelos se etiqueta BNC y se construye la matriz de confusi�n. Se realiza la misma operaci�n para cada mitad:

La tercer evaluaci�n de este experimento consiste en entrenar Stanford Tagger con un cuarto de WSJ y con un cuarto de WSJ + NFI. Posteriormente con estos dos modelos se etiqueta BNC y se construye la matriz de confusi�n. Se realiza la misma operaci�n para cada uno de los cuartos:

La cuarta evaluaci�n de este experimento consiste en entrenar Stanford Tagger con un d�cimo de WSJ y con un d�cimo de WSJ + NFI. Posteriormente con estos dos modelos se etiqueta BNC y se presentan los resultados.

En todos los casos (al igual que en los experimentos anteriores) se puede apreciar una leve mejora cuando se introduce NFI en el corpus de entrenamiento. Los datos y el detalle de estos experimentos se puede consultar en el ap�ndice.
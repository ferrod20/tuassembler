\section{Definiciones y marco te�rico}	
A continuaci�n se presentan definiciones y teor�as que ayudan a comprender el trabajo realizado. Se presenta el concepto de etiqueta gramatical, es decir, una etiqueta que identifica el rol que cumple una palabra dentro de cierto contexto. Se muestran los tipos de etiquetas que han sido utilizados intentando abarcar los distintos significados que pueden tener las palabras. Hasta el dia de hoy no se ha llegado a un consenso sobre un conjunto de etiquetas adecuado y se siguen explorando distintas alternativas.
Se explica el concepto de etiquetado gramatical, es decir, la tarea de asignar a cada palabra una etiqueta gramatical adecuada seg�n el contexto en donde �sta aparece. Se muestran ejemplos de que esta tarea est� muy lejos de ser trivial, introduciendo el concepto de ambig�edad gramatical. Esto ocurre cuando una palabra puede tener muchos significados (y por lo tanto distintas etiquetas gramaticales) dependiendo del contexto en d�nde aparece.

Se exhibe la importancia del etiquetado gramatical dentro de distintas �reas como la computaci�n ling�istica, reconocimiento y s�ntesis del habla. Se muestra como se maneja este proceso utilizando programas que lo realizan autom�ticamente, es decir, etiquetadores gramaticales autom�ticos. Se explica en profundidad como funcionan estos etiquetadores gramaticales autom�ticos, mostrando como las implementaciones actuales utilizan un proceso de entrenamiento. Este proceso ocurre a partir de un corpus previamente anotado que el etiquetador autom�tico toma como ejemplo para reproducir el etiquetado.
 
Se presenta el concepto de corpus y corpus anotados gramaticalmente como conjuntos de informaci�n extremadamente valiosos para todo estas tareas.
Se muestra la forma de medir, evaluar y comparar el rendimiento de los etiquetadores gramaticales, introduciendo los conceptos de corpus de entrenamiento y corpus de verificaci�n. Se muestran t�cnicas de an�lisis de error para el proceso de etiquetaci�n autom�tica. Se exhibe tambi�n el manejo de ciertos casos especiales dentro del proceso de etiquetaci�n autom�tico; las palabras desconocidas. Y por �ltimo se explican en detalle los etiquetadores autom�ticos utilizados en el presente trabajo.
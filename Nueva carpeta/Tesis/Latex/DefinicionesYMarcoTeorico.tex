\section{Definiciones y marco te�rico}	
A continuaci�n se presentan definiciones y teor�as sobre las que se basa el trabajo realizado. Se presenta el concepto de etiqueta gramatical, es decir, un c�digo que identifica el rol que cumple una palabra dentro de cierto contexto. Se muestran los conjuntos de etiquetas que han sido utilizados intentando abarcar los distintos significados que pueden tener las palabras. Se explica el concepto de etiquetado gramatical, es decir, la tarea de asignar a cada palabra una etiqueta gramatical adecuada seg�n el contexto en donde �sta aparece. Se muestran ejemplos de que esta tarea est� muy lejos de ser trivial, introduciendo el concepto de ambig�edad gramatical. 

Se exhibe la importancia del etiquetado gramatical dentro de distintas �reas como la computaci�n ling�istica, reconocimiento y s�ntesis del habla. Se muestra como se maneja este proceso utilizando programas que lo realizan autom�ticamente; los etiquetadores gramaticales autom�ticos. Se decriben implementaciones actuales que utilizan informaci�n estad�stica que el etiquetador utiliza para reproducir el etiquetado.
 
Se presenta el concepto de corpus y corpus anotado gramaticalmente como conjuntos de informaci�n extremadamente valiosos para todas estas tareas.
Se muestra la forma de medir, evaluar y comparar el rendimiento de los etiquetadores gramaticales, introduciendo los conceptos de corpus de entrenamiento y corpus de verificaci�n. Se muestran t�cnicas de an�lisis de error para la etiquetaci�n autom�tica. Se exhibe tambi�n el manejo de ciertos casos especiales dentro del proceso de etiquetaci�n autom�tico; las palabras desconocidas. Y por �ltimo se explican en detalle los etiquetadores autom�ticos utilizados en el presente trabajo.
\section{Traducci�n de etiquetas} 
Para cada una de sus definiciones, el diccionario Cobuild expone informaci�n gramatical expresada mediante etiquetas. Estas etiquetas gramaticales poseen un formato propio. Por ejemplo en la siguiente entrada de Cobuild para la palabra \textsl{canary}
{\small
\begin{verbatim}
DICTIONARY_ENTRY
k%en*!e*%eri
canary
canaries
A canary is a small yellow bird which sings beautifully. 
People sometimes keep canaries in cages as pets.  
countable noun
noun                                
\end{verbatim}
}

\noindent Se expone un ejemplo con informaci�n gramatical sobre la palabra:\\
\\
\textsl{People sometimes keep canaries in cages as pets.}  \\
\\
Se puede apreciar la etiqueta espec�fica \textsl{countable noun} asignada por Cobuild para canaries. Tambi�n se puede apreciar la etqueta general \textsl{noun} a la cual pertenece \textsl{countable noun}.\\

Como la idea de este trabajo es producir un corpus anotado a partir de este diccionario para utilizar como fuente de entrenamiento de etiquetadores gramaticales es necesario que el conjunto de etiquetas empleado sea el mismo que emplea el \textsl{Gold Standard} para poder medir posteriormente los resultados. Es por eso que se tom� la decisi�n de traducir estas etiquetas propias de Cobuild en etiquetas \textsl{Penn Treebank}, conjunto con el cual est� anotado el \textsl{Gold Standard}.
En esta traducci�n se gener� inevitablemente una p�rdida de informaci�n sem�ntica ya que las etiquetas \textsl{Penn Treebank} son menos espec�ficas que las etiquetas de Cobuild. 

El proceso de traducci�n de etiquetas intenta primero encontrar una traducci�n a la etiqueta espec�fica de Cobuild (en este ejemplo \textsl{countable noun}), si no fuera el caso, busca una traducci�n a la etiqueta general (\textsl{noun} para el ejemplo).
Esta decisi�n fu� tomada a partir de que se encontraron m�s de 3000 etiquetas espec�ficas diferentes, por lo tanto se decidi� crear una tabla de traducci�n de etiquetas s�lo para las etiquetas (generales y espec�ficas) que aparecen con mayor frecuencia. Con este m�todo logramos traducir aproximadamente el 99.26\% de las etiquetas.

A continuaci�n se presenta la tabla de traducci�n empleada:
\begin{center}
\begin{longtable}{| l | l |}
\caption{Tabla de traducci�n de etiquetas}\\	
	\hline		
\textbf{Etiqueta Cobuild}	&	\textbf{Etiqueta Penn Treebank}\\	
\hline
\endhead
\hline
\endfoot
\endlastfoot
	\hline
 coordinating conjunction & CC \\
number & CD \\
determiner & DT \\
determiner + countable noun in singular & DT \\
preposition & IN \\
subordinating conjunction & IN \\
preposition, or adverb after verb & IN \\
preposition after noun & IN \\
adjective & JJ \\
classifying adjective & JJ \\
qualitative adjective & JJ \\
adjective colour & JJ \\
ordinal & JJ \\
adjective after noun & JJ \\
modal & MD \\
adverb & RB \\
noun & NN \\
uncountable noun & NN \\
noun singular & NN \\
countable or uncountable noun & NN \\
countable noun with supporter & NN \\
uncountable or countable noun & NN \\
noun singular with determiner & NN \\
mass noun & NN \\
uncountable noun with supporter & NN \\
partitive noun & NN \\
noun singular with determiner with supporter & NN \\
countable noun + of & NN \\
countable noun, or by + noun & NN \\
countable noun or partitive noun & NN \\
count or uncountable noun & NN \\
countable noun or vocative & NN \\
partitive noun + uncountable noun & NN \\
noun singular with determiner + of & NN \\
noun in titles & NN \\
noun vocative & NN \\
uncountable noun + of & NN \\
indefinite pronoun & NN \\
uncountable noun, or noun singular & NN \\
countable noun, or in + noun & NN \\
partitive noun + noun in plural & NN \\
countable or uncountable noun with supporter & NN \\
uncountable noun, or noun before noun & NN \\
uncountable or countable noun with supporter & NN \\
noun before noun & NN \\
noun plural with supporter & NNP \\
noun in names & NNP \\
proper noun or vocative & NNP \\
proper noun & NNP \\
noun plural & NNS \\
predeterminer & PDT \\
pronoun & PP \\
possessive & PPS \\
adverb with verb & RB \\
adverb after verb & RB \\
sentence adverb & RB \\
adverb + adjective or adverb & RB \\
adverb + adjective & RB \\
preposition or adverb & RB \\
adverb after verb, or classifying adjective & RB \\
adverb or sentence adverb & RB \\
adverb with verb, or sentence adverb & RB \\
exclamation & UH \\
exclam & UH \\
verb & VB \\
verb + object & VB \\
verb or verb + object & VB \\
ergative verb & VB \\
verb + adjunct & VB \\
verb + object + adjunct & VB \\
verb + object (noun group or reflexive) & VB \\
verb + object or reporting clause & VB \\
verb + object (reflexive) & VB \\
verb + object, or phrasal verb & VB \\
verb + to-infinitive & VB \\
ergative verb + adjunct & VB \\
verb + object + adjunct (to) & VB \\
verb + object, or verb + adjunct & VB \\
verb + object + adjunct (with) & VB \\
verb + adjunct (with) & VB \\
verb + complement & VB \\
verb + object, or verb & VB \\
verb + object + to-infinitive & VB \\
verb + reporting clause & VB \\
verb or ergative verb & VB \\
verb + adjunct (from) & VB \\
wh: used as determiner & WDT \\
wh: used as relative pronoun & WP \\
wh: used as pronoun & WP \\
wh: used as adverb & WRB \\
	\hline
\end{longtable}
\end{center}